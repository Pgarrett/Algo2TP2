\section{Modulo Juego}

%TODO descripcion

\begin{Interfaz}

  \textbf{usa}: \tadNombre{Mapa, Coordenada}.

  \textbf{se explica con}: \tadNombre{Juego}.

  \textbf{generos}: \TipoVariable{juego}.

  \InterfazFuncion{CrearJuego}{\In{m}{mapa}}{juego}%
  [true]
  {$res$ \igobs crearJuego($m_0$) $\land$ mapa($res$) \igobs $m_0$}
  [O((largo(m) x ancho(m)) + copy(m))]
  [Crea el nuevo juego]

  \InterfazFuncion{AgregarPokemon}{\Inout{j}{juego}, \In{c}{coor}, \In{p}{pokemon}}{itConj}
  [$j \igobs j_0 \land puedoAgregarPokemon(c, j_0)$]
  {$j \igobs agregarPokemon(p, c, j_0)$}
  [$O(|P| + EC * log(EC))$]
  [EC es la maxima cantidad de jugadores esperando para atrapar un pokemon. |P| es el nombre mas largo para un pokemon en el juego]

  \InterfazFuncion{AgregarJugador}{\Inout{j}{juego}}{Nat}
  [$j \igobs j_0$]
  {$j \igobs agregarJugador(j_0) \land res = \#jugadores(j_0) + \#expulsados(j_0)$}
  [$O(J)$]
  [Agrega el jugador en el conjLineal, el iterador que devuelve el agregar se guarda en un vector donde la posicion es el id del jugador que voy a devolver]
  
  \InterfazFuncion{Conectarse}{\Inout{j}{juego}, \In{id}{Nat}, \In{c}{coor}}{}
  [$j \igobs j_0 \land id \in jugadores(j_0) \yluego \neg estaConectado(id, j_0) \land posExistente(c, mapa(j_0))$]
  {$j \igobs conectarse(id, c, j_0)$}
  [$O(log(EC))$]
  [Conecta al jugador pasado por parametro en la coordenada indicada]
  
  \InterfazFuncion{Desconectarse}{\Inout{j}{juego}, \In{id}{Nat}}{}
  [$j \igobs j_0 \land id \in jugadores(j_0) \yluego estaConectado(id, j_0)$]
  {$j \igobs desconectarse(id, j_0)$}
  [$O(log(EC))$]
  [Desconecta al jugador pasado por parametro]
  
  \InterfazFuncion{Moverse}{\Inout{j}{juego}, \In{id}{Nat}, \In{c}{coor}}{}
  [$j \igobs j_0 \land id \in jugadores(j_0) \yluego estaConectado(id, j_0) \land posExistente(c, mapa(j_0))$]
  {$j \igobs moverse(c, id, j_0)$}
  [$O((PS + PC) * |P| + log(EC))$]
  [Mueve al jugador pasado por parametro a la coordenada indicada]

  \InterfazFuncion{Mapa}{\In{j}{juego}}{Mapa}
  [true]
  {$res \igobs mapa(j)$}
  [$O(copy(mapa(j)))$]
  [Devuelve el mapa del juego]

  \InterfazFuncion{Jugadores}{\In{j}{juego}}{itConj(Jugador)}
  [true]
  {$res \igobs jugadores(j)$}
  [$\Theta(1)$]
  [Devuelve un iterador al conjunto de jugadores del juego]

  \InterfazFuncion{estaConectado}{\In{j}{juego}, \In{id}{Nat}}{Bool}
  [id $\in$ jugadores(j)]
  {$res \igobs estaConetado(id,j)$}
  [$\Theta(1)$]
  [Devuelve si el jugador con id ingresado esta conectado o no]

  \InterfazFuncion{posicion}{\In{j}{juego}, \In{id}{Nat}}{coor}
  [id $\in$ jugadores(j) $\yluego$ estaConectado(id,j)]
  {$res \igobs posicion(id,j)$}
  [$\Theta(1)$]
  [Devuelve la posicion actual del jugador con id ingresado si esta conectado]

  \InterfazFuncion{pokemones}{\In{j}{juego}, \In{id}{Nat}}{itConj(itDiccString)} %%Para borrar todos los pokemones a la hora de eliminar al jugador tenemos complejidad lineal, pero hace falta una estructura que admita repetidos ya que lo que devuelve es un multiconjunto, no queremos hacer un modulo multiconjunto queremos ser felices. Despues debatimos
  [id $\in$ jugadores(j)]
  {$res \igobs pokemons(id,j)$}
  [$\Theta(1)$]
  [Devuelve un iterador a la estructura que almacena los punteros a pokemons del jugador del id ingresado]

  \InterfazFuncion{expulsados}{\In{j}{juego}}{itConj(Jugador)}
  [True]
  {$res \igobs expulsados(j)$}
  [$\Theta(1)$]
  [Devuelve un iterador al conjunto de jugadores expulsados del juego]

  \InterfazFuncion{posConPokemones}{\In{j}{juego}}{itConj(Coor)}
  [True]
  {$res \igobs posConPokemons(j)$}
  [$O(1)$]
  [Devuelve un iterador al conjunto de coordenadas en donde hay pokemons]

  \InterfazFuncion{pokemonEnPos}{\In{j}{juego}, \In{c}{Coor}}{itPokemon}
  [c $\in$ posConPokemons(j)]
  {$res \igobs pokemonEnPos(c,j)$}
  [$\Theta(1)$] %Pero podemos tomarlos lo que queramos
  [Devuelve un iterador al pokemon de la coordenada dada]

  \InterfazFuncion{cantMovimientosParaCaptura}{\In{j}{juego}, \In{c}{Coor}}{Nat}
  [c $\in$ posConPokemons(j)]
  {$res \igobs cantMovimientosParaCaptura(c,j)$}
  [$\Theta(1)$] %Pero podemos tomarnos lo que queramos (?) no, enrealidad no.
  [Devuelve la cantidad de movimientos acumulados hasta el momento, para atrapar al pokemon de la coordenada dada]

  \InterfazFuncion{puedoAgregarPokemon}{\In{j}{juego}, \In{c}{Coor}}{Bool}
  [True]
  {$res \igobs puedoAgregarPokemon(c,j)$}
  [$\Theta\left(\displaystyle\sum_{c' \in coordendas(mapa(j))}equal(c,c')\right)$]
  [Devuelve si la coordenada ingresada es valida para agregar un pokemon en ella]

  \InterfazFuncion{hayPokemonCercano}{\In{j}{juego}, \In{c}{Coor}}{Bool}
  [True]
  {$res \igobs hayPokemonCercano(c,j)$}
  [$\Theta(1)$] 
  [Devuelve si la coordenada ingresada pertenece al rango de un pokemon salvaje]   

  \InterfazFuncion{posPokemonCercano}{\In{j}{juego}, \In{c}{Coor}}{Coor}
  [$hayPokemonCercano(c,j)$]
  {$res \igobs posPokemonCercano(c,j)$}
  [$\Theta(1)$] %Todas estas ultimas deberian ser O(1) gracias al magico DiccAcHashMagicTableListTupl pero por las dudas dejemo' la incognita hasta que sea oficial
  [Devuelve la coordenada mas del pokemon salvaje del rango siempre y cuando haya uno]   

  \InterfazFuncion{entrenadoresPosibles}{\In{c}{coor}, \In{es}{conjLineal(jugador)}, \In{j}{juego}}{itColaPrior(itJugador)}
  [$hayPokemonCercano(c,j) \yluego pokemonEnPos(posPokemonCercano(c,j),j).jugadoresEnRango \subseteq jugadoresConectados(c,j)$]
  {$res \igobs entrenadoresPosibles(c,pokemonEnPos(posPokemonCercano(c,j),j).jugadoresEnRango,j)$}
  [$O(Cardinal(es))$] 
  [Devuelve un iterador a los jugadores que estan esperando para atrapar al pokemon mas cercano a la coordenada ingresada]   

  \InterfazFuncion{indiceRareza}{\In{j}{juego}, \In{p}{Pokemon}}{Nat}
  [$p \in todosLosPokemons(j)$]
  {$res \igobs indiceRareza(p,j)$}
  [$O(|P|)$] 
  [Devuelve el indice de rareza del pokemon del juego ingresado]

  \InterfazFuncion{cantPokemonesTotales}{\In{j}{juego}}{Nat}
  [true]
  {$res \igobs cantPokemonsTotales(p)$}
  [$\Theta(1)$] 
  [Devuelve la cantidad de pokemones que hay en el juego]

  \InterfazFuncion{cantMismaEspecie}{\In{j}{juego}, \In{p}{Pokemon}}{Nat}
  [true]
  {$res \igobs cantMismaEspecie(p, pokemons(j), j$}
  [$O(|P|)$] 
  [Devuelve la cantidad de pokemones de la especie ingresada hay en el juego]

  
\end{Interfaz}

\begin{Representacion}
\subsubsection{Representación de Juego}
	\begin{Estructura}{Juego}[estr]
		\begin{Tupla}[estr]
			\tupItem{pokemones}{diccString(String, ListaPorTipo(itConj(infoPokemon)))}%
			\tupItem{todosLosPokemones}{conjLineal(infoPokemon)}
			\tupItem{jugadores}{conjLineal(infoJugador)}
			\tupItem{expulsados}{conjLineal(Nat)}
			\tupItem{jugadoresPorID}{Vector($<$itConj(infoJugador), itColaPrior(jugador)$>$)} %% itConj es un iterador a jugadores, itColaPrior es un iterador al heap en el que esta el jugador
			\tupItem{posicionesPokemons}{DiccMat(coor, itConj(infoPokemon))} %% itConj es un iterador al conjunto de todosLosPokemones
			\tupItem{posicionesJugadores}{DiccMat(coor, conjLineal(Nat))} %% cada coordenada tiene un conjunto de jugadores. con el jugador se lo puede busacr en jugadoresPorID, donde la primer coordenada se agrega al heap del pokemon, lo que devuelve el agregar al heap es el segundo valor en la tupla
			\tupItem{mapa}{Mapa}
		\end{Tupla}
	\end{Estructura}
	
		\begin{Tupla}[infoJugador]
			\tupItem{estaConectado}{Bool}
			\tupItem{sanciones}{Nat}
			\tupItem{pokeCapturados}{conjLineal(itConj(infoPokemon))} %% Revisar si es infoPokemon o pokemones. tiene que apuntar al conjunto de todos los pokemones
		\end{Tupla}
	
		\begin{Tupla}[infoPokemon]
			\tupItem{contador}{Nat}
			\tupItem{jugadoresEnRango}{colaPrior$<$Nat, itConj(jugadores)$>$}
			\tupItem{salvaje}{Bool}
		\end{Tupla}
	

\subsubsection{Invariante de Representación}
	\begin{enumerate}
		\item La suma de todos los significados de pokemones es igual al cardinales de  todosLosPokemones.
		\item La suma de la cantidad de jugadores y expulsados es igual a la longitud del vector jugadoresPorID.
		\item Para toda coordenada, si esta definida en posicionesPokemons entonces la coordeanda pertenece al mapa.
		\item La posicion de todo jugador que pertenezca al conjunto jugadores y este conectado pertenece al mapa.
		\item Para todo pokemon que exista en pokemons y sea salvaje, el conjunto de jugadores que esta esperando para atraparlo pertenece al conjunto jugadores.
		\item Todo jugador que pertenezca a jugadores, este conectado y este esperando para atrapar, esta incluido en el conjunto de jugadores en rango del pokemon al que quiere atrapar.
		\item Los conjuntos jugadores y expulsados son disjuntos.
	\end{enumerate}
	
	\begin{enumerate}
		\item Checkear con significado de trie
		\item \# e.jugadores + \# e.expulsados = long(e.jugadoresPorID)
		\item ($\forall c : coor$) def?(c, e.posicionesPokemons) $\impluego$ j.posicion $\in$ e.mapa.coordenadas
		\item ($\forall j : jug$) j $\in$ e.jugadores $\land$ j.estaConectado $\impluego$ j.posicion $\in$ e.mapa.coordenadas
		\item ($\forall p : poke$) (def?(p, e.pokemones) $\land$ p.salvaje) $\impluego$ ($\forall it : itJug$) HayMas?(it) $\yluego$ Actual(it) $\in$ p.jugadoresEnRango $\impluego$ Actual(it) $\in$ e.jugadores
		\item ($\forall j : jug$) j $\in$ e.jugadores $\land$ j.estaConectado $\yluego$ estaParaAtrapar(j) $\impluego$ ($\forall p : poke$) def?(p, e.pokemones) $\yluego$ j $\in$ p.jugadoresEnRango
		\item ($\forall j : jug$) (j $\in$ e.jugadores $\impluego$ j $\notin$ e.expulsados) $\lor$ (j $\in$ e.expulsados $\impluego$ j $\notin$ e.jugadores)
		
	\end{enumerate}
	
\subsubsection{Funci\'on de Abstracci\'on}	
<<<<<<< HEAD
%\end{Representacion}
	$Abs(e): estre - > Jugo {Rep(e)} 
 pGo: Juego tq e.mapa = mapa(pGo) y e.jugadores = jugadores(pGo) yluego \\
 (Para todo j : jugador) j pertenece e.jugadores impluego 
  j.sanciones = sanciones(j, pGo) ((j pertenece expulsados(pGo) y j.sanciones >= 10)
 oluego (j.pokesCapturados = pokemones(j,pGo) y j.estaConectado = estaConectad(j,pGo) \\
 y j.estaConectado impluego j.pos = posicion(j,pGo))) y \\
 (Para todo p : pokemon) p pertenece c.pokemones impluego (Para todo j : Jugador) \\
 j pertenece e.jugadores yluego p pertenece pokemones(j,pGo) o [(Para todo c : coord)\\
 c pertenece e.mapa.coordenadas yluego p = pokemonEnPos(c,pGo) y cantMovParaCap(c,pGo)\\
 p.contador]$
=======
%	Abs(e): estre - $>$ Jugo {Rep(e)} 
% pGo: Juego tq e.mapa = mapa(pGo) y e.jugadores = jugadores(pGo) yluego \\
% (Para todo j : jugador) j pertenece e.jugadores impluego 
% \\ j.sanciones = sanciones(j, pGo) ((j pertenece expulsados(pGo) y j.sanciones >= 10)\\
% oluego (j.pokesCapturados = pokemones(j,pGo) y j.estaConectado = estaConectad(j,pGo) \\
% y j.estaConectado impluego j.pos = posicion(j,pGo))) y \\
% (Para todo p : pokemon) p pertenece c.pokemones impluego (Para todo j : Jugador) \\
% j pertenece e.jugadores yluego p pertenece pokemones(j,pGo) o [(Para todo c : coord)\\
% c pertenece e.mapa.coordenadas yluego p = pokemonEnPos(c,pGo) y cantMovParaCap(c,pGo)\\
% p.contador]
>>>>>>> 459f9ae7cc6890764848afad8ccb3cf080929eb1
\end{Representacion}

%%% Sanciones	O(1)
%%% Contador	O(1)
%%% Heap		O(log(EC))
%%% Atrapar		O(PS + PC)
%%% BorrarJ		O(|P| * PC)

\subsection{Algoritmos}

\begin{algorithm}[H]{\textbf{iCrearJuego}(\In{m}{Mapa)}) $\to$ $res$ : $Juego$}
	\begin{algorithmic}
		\State $Juego : j$	\Comment $O(1)$
		\State j.pokemones $\gets$ CrearDiccionario()	\Comment $O(1)$
		\State j.todosLosPokemones $\gets$ Vacio()	\Comment $O(1)$
		\State j.jugadores $\gets$ Vacio()	\Comment $O(1)$
		\State j.expulsados $\gets$ Vacio()	\Comment $O(1)$
		\State j.jugadoresPorID $\gets$ Vacia()	\Comment $O(1)$
		\State j.posicionesPokemons $\gets$	Vacio(largo(m), ancho(m))	\Comment O(largo(m) x ancho(m))
		\State j.mapa $\gets$ m	\Comment $O(copy(m))$

		\State $res \gets j$ \Comment $O(1)$
		\medskip
		\Statex \underline{Complejidad:} O((largo(m) x ancho(m)) + copy(m))
		\Statex \underline{Justificación:} 

    \end{algorithmic}
\end{algorithm}

\begin{algorithm}[H]{\textbf{iAgregarPokemon}(\Inout{j}{juego)}, \In{c}{coor)}, \In{p}{pokemon)}) $\to$ $res$ : $itPokemon$}
	\begin{algorithmic}
	
	\State $ItColaPrior(itJugador) it \gets entrenadoresPosibles(j,c) $ \Comment $O(|1|)$
	\While {it.HaySiguiente()} \Comment $O(1)$
		\State $Definir(p.jugadoresEnRango, it.Siguiente().id, it.Siguiente()) $ \Comment $O(log(|entrenadoresPosibles|))$
	\EndWhile
	\State $p.salvaje \gets TRUE $ \Comment $O(|1|)$
	\State $p.contador \gets 0 $ \Comment $O(|1|)$
	\State $j.pokemonsTotales \gets j.pokemonsTotales+1 $ \Comment $O(|1|)$
	
	\If {Definido?(pokemones,p.tipo)} \Comment $O(|P|)$
		\State ItPokemon poke $\gets$ Obtener(j.pokemones, p.tipo)  \Comment $O(|p.tipo|)$
	\Else
		\State ItPokemon poke $\gets$ Definir(j.pokemones, p.tipo, p) \Comment $O(|p.tipo|)$
	\EndIf
	\State $res \gets Definir(j.posicionesPokemon, coord, <poke,true>) $ \Comment $O(|1|)$
		\medskip
		\Statex \underline{Complejidad:} $O(|p.tipo| + |entrenadoresPosibles| * log(|entrenadoresPosibles|))$
		 esta mal creo, pero no se que meterle
		\Statex \underline{Justificación:}  definir, preguntar si esta definido 
		y obtener el pokemon son la longitud del tipo ya que representan una insercion 
		o busqueda en un trie, el ciclo recorre todos los entrenadores posibles, 
		los cuales pertenecen a un conjunto acotado por el rango del pokemon, 
		hay tantos ciclos como entrenadores posibles y por cada uno de ellos 
		hay que definirlo en un heap 
    \end{algorithmic}
\end{algorithm}

\begin{algorithm}[H]{\textbf{iAgregarJugador}(\Inout{j}{juego)}) $\to$ $res$ : $Nat$}
	\begin{algorithmic}
	\State Jugador: jug \Comment $O(1)$
	\State $jug.id \gets Cardinal(j.jugadores) + Cardinal(j.expulsados)$ \Comment $O(1)$
	\State $jug.estaConectado \gets false $ \Comment $O(1)$
	\State $jug.sanciones \gets 0 $ \Comment $O(1)$	
	\State $jug.pokeCapturados \gets Vacio() $ \Comment $O(1)$
	\State $jug.posicion \gets NULL $ \Comment $O(1)$
	\State $itJ \gets AgregarRapido(j.jugadores,jug) $ \Comment $O(copy(jug))$
	\State $AgregarAtras(j.jugadoresPorID, <itJ, NULL>)$ \Comment $O(J)$ {Donde J es la cantidad total de jugadores que fueron agregados al juego}
	\State $res \gets jug.id $ \Comment $O(1)$
	
		\medskip
		\Statex \underline{Complejidad:} $O(J)$
		\Statex \underline{Justificación:} O(copy(jug)) es igual a O(1) ya que solamente es copiar Nat, Bool y un conjunto vacio.
    \end{algorithmic}
\end{algorithm}

\begin{algorithm}[H]{\textbf{iConectarse}(\Inout{j}{juego}, \In{e}{Nat}, \In{c}{Coor})}
	\begin{algorithmic}
		\State tupJug $\gets$ j.jugadoresPorId[e]	\Comment $O(1)$
		\State itJug $\gets$ tupJug[1]	\Comment $O(1)$
		\State jug $\gets$ Siguiente(itJug)	\Comment $O(1)$
		\State jug.estaConectado $\gets$ true	\Comment $O(1)$
		\State jug.posicion $\gets$ c	\Comment $O(1)$
		\If {HayPokemonCercano(j, c)}	\Comment $O(1)$
			\State p $\gets$ Siguiente(Significado(j.posicionesPokemons, PosPokemonCercano(j, c)))	\Comment $O(1)$
			\State tupJug[2] $\gets$ Encolar(p.jugadoresEnRango, Cardinal(jug.pokeCapturados), itJug)	\Comment $O(log(EC))$
			\State p.contador $\gets$ 0	\Comment $O(1)$
		\EndIf
	
		\medskip
		\Statex \underline{Complejidad:} $O(log(EC))$
		\Statex \underline{Justificación:} EC es la maxima cantidad de jugadores esperando para atrapar un pokemon. En el peor caso, el heap al que entra el jugador es el que mas jugadores esperando tiene.
    \end{algorithmic}
\end{algorithm}

\begin{algorithm}[H]{\textbf{iDesconectarse}(\Inout{j}{juego)}, \In{id}{Nat)}} 
	\begin{algorithmic}
		\State tupJug $\gets$ j.jugadoresPorId[e]	\Comment $O(1)$
		\If {HayPokemonCercano(j, c)}	\Comment $O(1)$
			\State tupJug[2] $\gets$ EliminarSiguiente(tupJug[2])	\Comment $O(log(EC))$
		\EndIf
		\State Siguiente(tupJug[1]).estaConectado $\gets$ false	\Comment $O(1)$
	
		\medskip
		\Statex \underline{Complejidad:} $O(log(EC))$
		\Statex \underline{Justificación:} EC es la maxima cantidad de jugadores esperando para atrapar un pokemon. En el peor caso, el heap del que sale el jugador es el que mas jugadores esperando tiene.
    \end{algorithmic}
\end{algorithm}

\begin{algorithm}[H]{\textbf{Moverse}(\Inout{j}{juego}, \In{id}{Nat}, \In{c}{coor})}{}
	\begin{algorithmic}
	\If {debeSancionarse$?$(Siguiente(jugadoresPorID[id]), c, j)} \Comment $O(|P|)$
		\If {$campo_1(Siguiente(jugadoresPorID[id])).sanciones$ < 4}
			\State $campo_1(jugadoresPorID[id]->eliminarSiguiente)$
			\If {hayPokemonCercano(j, c)}
				\State $(PokemonEnPos \rightarrow siguiente).jugadoresEnRango.Eliminar(campo_2(jugadoresPorID[id])$
				\State $campo_2(jugadoresPorID[id]->eliminarSiguiente)$
			\Else
				\State $campo_1(jugadoresPorID[id]->siguiente).sanciones$ $\gets$ $campo_1(jugadoresPorID[id]->siguiente).sanciones$ + 1
			\EndIf
		\EndIf
	\EndIf

\medskip
\Statex \underline{Complejidad:} $O()$
\Statex \underline{Justificación:}
\end{algorithmic}
\end{algorithm}

\begin{algorithm}[H]{\textbf{iDebeSancionarse}(\In{e}{jugador}, \In{c}{coor}, \In{j}{juego}) $\to$ $res$ : $Bool$} 
\\
	{$\textbf{Pre}$ $\equiv$ e $\in$ Jugadores(j)}
	\begin{algorithmic}
		\State $res \gets \neg HayCamino(e.posicion, c, j.mapa) \lor DistEuclidea(e.posicion, c, mapa) > 100 $ \Comment $O(1)$
	
		\medskip
		\Statex \underline{Complejidad:} $O(1)$
		\Statex \underline{Justificación:} Checkea si el jugador hizo un movimiento invalido.
    \end{algorithmic}
\end{algorithm}

\begin{algorithm}[H]{\textbf{iMapa}(\In{j}{juego}) $\to$ $res$ : $Mapa$} 
	{}
	\begin{algorithmic}
		\State $res \gets j.mapa $ \Comment $O(copy(mapa(j)))$
	
		\medskip
		\Statex \underline{Complejidad:} $O(copy(mapa(j)))$
		\Statex \underline{Justificación:} Devuelve el mapa del juego por copia.
    \end{algorithmic}
\end{algorithm}

\begin{algorithm}[H]{\textbf{iJugadores}(\In{j}{juego}) $\to$ $res$ : $itConj(jugador)$} 
	{}
	\begin{algorithmic}
		\State $res \gets CrearIt(j.jugadores) $ \Comment $O(1)$
	
		\medskip
		\Statex \underline{Complejidad:} $O(1)$
		\Statex \underline{Justificación:} Devuelve el mapa del juego.
    \end{algorithmic}
\end{algorithm}

\begin{algorithm}[H]{\textbf{iEstaConectado}(\In{j}{juego}, \In{id}{Nat}) $\to$ $res$ : Bool} 
	{}
	\begin{algorithmic}
		\State $res \gets Siguiente(j.jugadoresPorID[id]_0).estaConectado $ \Comment $O(1)$
	
		\medskip
		\Statex \underline{Complejidad:} $O(1)$
		\Statex \underline{Justificación:} Devuelve si el jugador esta conectado.
    \end{algorithmic}
\end{algorithm}

\begin{algorithm}[H]{\textbf{iPosicion}(\In{j}{juego}, \In{id}{Nat}) $\to$ $res$ : coor} 
	{}
	\begin{algorithmic}
		\State $res \gets Siguiente(j.jugadoresPorID[id]_0).posicion $ \Comment $O(1)$
	
		\medskip
		\Statex \underline{Complejidad:} $O(1)$
		\Statex \underline{Justificación:} Devuelve si el jugador esta conectado.
    \end{algorithmic}
\end{algorithm}

\begin{algorithm}[H]{\textbf{iPokemones}(\In{j}{juego}, \In{id}{Nat}) $\to$ $res$ : itConj(itDiccString)} 
	{}
	\begin{algorithmic}
		\State $res \gets CrearIt(Siguiente(j.jugadoresPorID[id]_0).pokeCapturados) $ \Comment $O(1)$
	
		\medskip
		\Statex \underline{Complejidad:} $O(1)$
		\Statex \underline{Justificación:} Devuelve un iterador al conjunto de pokemones atrapados por el jugador.
    \end{algorithmic}
\end{algorithm}

\begin{algorithm}[H]{\textbf{iExpulsados}(\In{j}{juego}) $\to$ $res$ : $itConj(Jugador)$} 
	{}
	\begin{algorithmic}
		\State $res \gets CrearIt(j.expulsados) $ \Comment $O(1)$
	
		\medskip
		\Statex \underline{Complejidad:} $O(1)$
		\Statex \underline{Justificación:} Devuelve un iterador al conjunto de jugadores expulsados.
    \end{algorithmic}
\end{algorithm}

\begin{algorithm}[H]{\textbf{iPosConPokemones}(\In{j}{juego}) $\to$ $res$ : $itConj(Coor)$} 
	{}
	\begin{algorithmic}
		\State $res \gets CrearIt(Coordenadas(j.posicionesPokemons)) $ \Comment $O(1)$
	
		\medskip
		\Statex \underline{Complejidad:} $O(1)$
		\Statex \underline{Justificación:} Devuelve las posiciones con pokemones.
    \end{algorithmic}
\end{algorithm}

\begin{algorithm}[H]{\textbf{iPokemonEnPos}(\In{j}{juego}, \In{c}{coor}) $\to$ $res$ : $itConj(Pokemon)$} 
	{}
	\begin{algorithmic}
		\State $res \gets Significado(j.posicionesPokemons, c) $ \Comment $O(1)$
	
		\medskip
		\Statex \underline{Complejidad:} $O(1)$
		\Statex \underline{Justificación:} Devuelve el mapa del juego.
    \end{algorithmic}
\end{algorithm}

\begin{algorithm}[H]{\textbf{iCantMovimientosParaCaptura}(\In{j}{juego}, \In{c}{coor}) $\to$ $res$ : $Nat$} 
	{}
	\begin{algorithmic}
		\State res $\gets$ 10 - Siguiente(Significado(j.posicionesPokemons, c)).contador  \Comment $O(1)$
	
		\medskip
		\Statex \underline{Complejidad:} $O(1)$
		\Statex \underline{Justificación:} Devuelve cuantos movimientos faltan para capturar al pokemon.
    \end{algorithmic}
\end{algorithm}

\begin{algorithm}[H]{\textbf{iPosPokemonCercano}(\In{j}{juego}, \In{c}{Coor})  $\to$ $res$ : $coor$} 
	{}
	\begin{algorithmic}
		\State i $\gets$ 0 \Comment $O(1)$
		\State latC $\gets$ Latitud(c)	\Comment $O(1)$
		\State i $\gets$ DamePos(latC, 2) \Comment $O(1)$		
		\State longC $\gets$ Longitud(c)	\Comment $O(1)$
		\State j $\gets$ DamePos(longC, 2) \Comment $O(1)$
		\While {i $<$ latC + 2}	\Comment $O(1)$ {Vale porque estoy recorriendo un conjunto acotado de coordenadas}
			\While {j $<$ longC + 2}	\Comment $O(1)$ {Vale porque estoy recorriendo un conjunto acotado de coordenadas}
				\If {Definido?(j.posicionesPokemons, $<$i, j$>$) $\land$ DistEuclidea(c, $<$i, j$>$) $\leq$ 4}	\Comment $O(1)$
					\State res $\gets$ $<$i, j$>$	\Comment $O(1)$ 
				\EndIf
				\State j $\gets$ j + 1	\Comment $O(1)$
			\EndWhile
			\State i $\gets$ i + 1	\Comment $O(1)$
		\EndWhile
		\medskip
		\Statex \underline{Complejidad:} $O(1)$
		\Statex \underline{Justificación:} Como el rango a recorrer es una constante, se puede decir que es de la clase $\Theta(1)$ 
    \end{algorithmic}
\end{algorithm}

\begin{algorithm}[H]{\textbf{iPuedoAgregarPokemon}(\In{j}{juego}, \In{c}{coor}) $\to$ $res$ : $bool$} 
	\begin{algorithmic}
		\State res $\gets$ PosExistente(c, j.mapa) $\land$ $\lnot$(Definido?(j.posicionesPokemons, c)) $\land$ $\lnot$(HayPokemonCercano(j, c))  \Comment $\Theta\left(\displaystyle\sum_{c' \in coordendas(mapa(j))}equal(c,c')\right)$
	
		\medskip
		\Statex \underline{Complejidad:} [$\Theta\left(\displaystyle\sum_{c' \in coordendas(mapa(j))}equal(c,c')\right)$]
		\Statex \underline{Justificación:} Tiene que ver si la posicion existe en el mapa, las demas operaciones son O(1)
     \end{algorithmic}
 \end{algorithm}
 
 \begin{algorithm}[H]{\textbf{HayPokemonCercano}(\In{j}{juego}, \In{c}{coor}) $\to$ $res$ : $bool$} 
	\begin{algorithmic}
		\State res $\gets$ false	\Comment $O(1)$ 
		\State latC $\gets$ Latitud(c)	\Comment $O(1)$
		\State i $\gets$ DamePos(latC, 5) \Comment $O(1)$		
		\State longC $\gets$ Longitud(c)	\Comment $O(1)$
		\State j $\gets$ DamePos(longC, 5) \Comment $O(1)$
		\While {i $<$ latC + 5}	\Comment $O(1)$ {Vale porque estoy recorriendo un conjunto acotado de coordenadas}
			\While {j $<$ longC + 5 }	\Comment $O(1)$ {Vale porque estoy recorriendo un conjunto acotado de coordenadas}
				\If {Definido?(j.posicionesPokemons, $<$i, j$>$) $\land$ DistEuclidea(c, $<$i, j$>$) $\leq$ 25}	\Comment $O(1)$
					\State res $\gets$ true	\Comment $O(1)$ 
				\EndIf
				\State j $\gets$ j + 1	\Comment $O(1)$
			\EndWhile
			\State i $\gets$ i + 1	\Comment $O(1)$
		\EndWhile
	
		\medskip
		\Statex \underline{Complejidad:} $\Theta(1)$
		\Statex \underline{Justificación:} Como el rango a recorrer es una constante, se puede decir que es de la clase $\Theta(1)$
     \end{algorithmic}
 \end{algorithm}

\begin{algorithm}[H]{\textbf{iEntrenadoresPosibles}(\In{c}{coor}, \In{es}{conjLineal(jugador)}, \In{j}{juego})  $\to$ $res$ : $conjLineal(itConj)$} 
	{}
	\begin{algorithmic}
		\State ePosibles $\gets$ Vacia()  \Comment $O(1)$ {Crea un conjunto de iteradores vacio}
		\If {Cardinal(es) $!=$ 0}	\Comment $O(1)$
			\State itE $\gets$ CrearIt(es)	\Comment $O(1)$
			\While {HaySiguiente(itE)} \Comment $O(Cardinal(es))$ {Es la cantidad de jugadores que haya en el conjunto es}
				\State posJugador $\gets$ Siguiente(itE).posicion	\Comment $O(1)$
				\If {(iHayPokemonCercano(posJugador, j) $\yluego$ \Comment $O(1)$
				 \State iPosPokemonCercano(posJugador, j) == c $\land$ \Comment $O(1)$
				 \State iHayCamino(c, posJugador, Mapa(j)))} \Comment $O(1)$
					\State AgregarRapido(ePosibles, Siguiente(itE))	\Comment $O(1)$ {Copiar un iterador es O(1)}
				\EndIf
				\State Avanzar(itE)	\Comment $O(1)$
			\EndWhile
		\EndIf
		\State $res \gets ePosibles$ \Comment $O(1)$
	
		\medskip
		\Statex \underline{Complejidad:} $O(Cardinal(es))$
		\Statex \underline{Justificación:} Se itera por completo el conjunto de jugadores 'es'. En peor caso, todos los elementos de 'es' deben ser agregados al resultado.
    \end{algorithmic}
\end{algorithm}

\begin{algorithm}[H]{\textbf{iIndiceRareza}(\In{j}{juego}, \In{p}{pokemon})  $\to$ $res$ : $Nat$} 
	{}
	\begin{algorithmic}
		\State $cuantosP \gets iCantMismaEspecie(j, p) $ \Comment $O(|p.tipo|)$
		\State res $\gets$ 100 - (100 x cuantosP / iCantPokemonesTotales)  \Comment $O(1)$
	
		\medskip
		\Statex \underline{Complejidad:} $O(|P|)$
		\Statex \underline{Justificación:} Siendo $|P|$ el nombre mas largo para un pokemon en el juego.
    \end{algorithmic}
\end{algorithm}

\begin{algorithm}[H]{\textbf{iCantPokemonesTotales}(\In{j}{juego}) $\to$ $res$ : $Nat$} 
	{}
	\begin{algorithmic}
		\State $res \gets cardinal(j.todosLosPokemones) $ \Comment $O(1)$
	
		\medskip
		\Statex \underline{Complejidad:} $O(1)$
		\Statex \underline{Justificación:} Pide el cardinal de un conjunto.
    \end{algorithmic}
\end{algorithm}

\begin{algorithm}[H]{\textbf{iCantMismaEspecie}(\In{j}{juego}, \In{p}{Pokemon}) $\to$ $res$ : $Nat$} 
	{}
	\begin{algorithmic}
		\If {Definido?(j.pokemones, p.tipo)} \Comment $O(|P|)$
		\State $res \gets Obtener(j.pokemones, p.tipo)$ \Comment $O(|P|)$
		\Else
		\State $res \gets 0 $ \Comment $O(1)$
		\EndIf
	
		\medskip
		\Statex \underline{Complejidad:} $O(|P|)$
		\Statex \underline{Justificación:} En peor caso, el pokemon que se busca es el de nombre mas largo o no esta en el diccionario.
    \end{algorithmic}
\end{algorithm}

\begin{algorithm}[H]{\textbf{DamePos}(\In{p}{Nat}, \In{step}{Nat}) $\to$ $res$ : $Nat$} 
	{}
	\begin{algorithmic}
		\State i $\gets$ p	\Comment $O(1)$
		\State fin $\gets$ false	\Comment $O(1)$
		\State $res \gets i$	\Comment $O(1)$
		\While {i $\neq$ 1 $\land \neg$ fin}	\Comment $O(1)$
			\State i $\gets$ i - 1	\Comment $O(1)$
			\If {i == p - step}	\Comment $O(1)$
				\State fin $\gets$ true	\Comment $O(1)$
			\EndIf
		\EndWhile
	
		\medskip
		\Statex \underline{Complejidad:} $O(1)$
		\Statex \underline{Justificación:} Recorre un rango acotado.
    \end{algorithmic}
\end{algorithm}