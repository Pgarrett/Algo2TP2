\section{Modulo Juego}

%TODO descripcion

\begin{Interfaz}

  \textbf{usa}: \tadNombre{Mapa, Coordenada}.

  \textbf{se explica con}: \tadNombre{Juego}.

  \textbf{generos}: \TipoVariable{juego}.

  \InterfazFuncion{CrearJuego}{\In{m}{map}}{juego}%
  [true]
  {$res$ \igobs crearJuego($m_0$) $\land$ mapa($res$) \igobs $m_0$}
  [O((largo(m) x ancho(m)) + copy(m))]
  [Crea el nuevo juego]

  \InterfazFuncion{AgregarPokemon}{\Inout{j}{juego}, \In{c}{coor}, \In{p}{Pokemon}}{itConj(Pokemon)}
  [$j \igobs j_0 \land puedoAgregarPokemon(c, j_0)$]
  {$j \igobs agregarPokemon(p, c, j_0)$}
  [$O(|P| + EC * log(EC))$]
  [EC es la maxima cantidad de jugadores esperando para atrapar un pokemon. |P| es el nombre mas largo para un pokemon en el juego]

  \InterfazFuncion{AgregarJugador}{\Inout{j}{juego}}{Jugador}
  [$j \igobs j_0$]
  {$j \igobs agregarJugador(j_0) \land res = \#jugadores(j_0) + \#expulsados(j_0)$}
  [$O(J)$]
  [Agrega el jugador en el conjLineal, y devuelve su identificador]
  
  \InterfazFuncion{Conectarse}{\Inout{j}{juego}, \In{id}{Jugador}, \In{c}{coor}}{}
  [$j \igobs j_0 \land id \in jugadores(j_0) \yluego \neg estaConectado(id, j_0) \land posExistente(c, mapa(j_0))$]
  {$j \igobs conectarse(id, c, j_0)$}
  [$O(log(EC))$]
  [Conecta al jugador pasado por parametro en la coordenada indicada]
  
  \InterfazFuncion{Desconectarse}{\Inout{j}{juego}, \In{id}{Jugador}}{}
  [$j \igobs j_0 \land id \in jugadores(j_0) \yluego estaConectado(id, j_0)$]
  {$j \igobs desconectarse(id, j_0)$}
  [$O(log(EC))$]
  [Desconecta al jugador pasado por parametro]
  
  \InterfazFuncion{Moverse}{\Inout{j}{juego}, \In{id}{Jugador}, \In{c}{coor}}{}
  [$j \igobs j_0 \land id \in jugadores(j_0) \yluego estaConectado(id, j_0) \land posExistente(c, mapa(j_0))$]
  {$j \igobs moverse(c, id, j_0)$}
  [$O((PS + PC) * |P| + log(EC))$]
  [Mueve al jugador pasado por parametro a la coordenada indicada]

  \InterfazFuncion{Mapa}{\In{j}{juego}}{map}
  [true]
  {$res \igobs mapa(j)$}
  [$O(copy(mapa(j)))$]
  [Devuelve el mapa del juego]

  \InterfazFuncion{Jugadores}{\In{j}{juego}}{itConj(Jugador)}
  [true]
  {$res \igobs jugadores(j)$}
  [$\Theta(1)$]
  [Devuelve un iterador al conjunto de jugadores del juego]

  \InterfazFuncion{estaConectado}{\In{j}{juego}, \In{id}{Jugador}}{Bool}
  [id $\in$ jugadores(j)]
  {$res \igobs estaConetado(id,j)$}
  [$\Theta(1)$]
  [Devuelve si el jugador con id ingresado esta conectado o no]

  \InterfazFuncion{posicion}{\In{j}{juego}, \In{id}{Jugador}}{coor}
  [id $\in$ jugadores(j) $\yluego$ estaConectado(id,j)]
  {$res \igobs posicion(id,j)$}
  [$\Theta(1)$]
  [Devuelve la posicion actual del jugador con id ingresado si esta conectado]

  \InterfazFuncion{pokemones}{\In{j}{juego}, \In{id}{Jugador}}{itConj($<$Pokemon, Nat$>$)} %%Para borrar todos los pokemones a la hora de eliminar al jugador tenemos complejidad lineal, pero hace falta una estructura que admita repetidos ya que lo que devuelve es un multiconjunto, no queremos hacer un modulo multiconjunto queremos ser felices. Despues debatimos
  [id $\in$ jugadores(j)]
  {$res \igobs pokemons(id,j)$}
  [$\Theta(1)$]
  [Devuelve un iterador a la estructura que almacena los punteros a pokemons del jugador del id ingresado]

  \InterfazFuncion{expulsados}{\In{j}{juego}}{itConj(Jugador)}
  [True]
  {$res \igobs expulsados(j)$}
  [$\Theta(1)$]
  [Devuelve un iterador al conjunto de jugadores expulsados del juego]

  \InterfazFuncion{posConPokemones}{\In{j}{juego}}{itConj(coor)}
  [True]
  {$res \igobs posConPokemons(j)$}
  [O(1)]
  [Devuelve un iterador al conjunto de coordenadas en donde hay pokemons]

  \InterfazFuncion{pokemonEnPos}{\In{j}{juego}, \In{c}{coor}}{Pokemon}
  [c $\in$ posConPokemons(j)]
  {$res \igobs pokemonEnPos(c,j)$}
  [$\Theta(1)$] %Pero podemos tomarlos lo que queramos
  [Devuelve un iterador al pokemon de la coordenada dada]

  \InterfazFuncion{cantMovimientosParaCaptura}{\In{j}{juego}, \In{c}{coor}}{Nat}
  [c $\in$ posConPokemons(j)]
  {$res \igobs cantMovimientosParaCaptura(c,j)$}
  [$\Theta(1)$] %Pero podemos tomarnos lo que queramos (?) no, enrealidad no.
  [Devuelve la cantidad de movimientos acumulados hasta el momento, para atrapar al pokemon de la coordenada dada]

  \InterfazFuncion{puedoAgregarPokemon}{\In{j}{juego}, \In{c}{coor}}{Bool}
  [True]
  {$res \igobs puedoAgregarPokemon(c,j)$}
  [$\Theta\left(\displaystyle\sum_{c' \in coordendas(mapa(j))}equal(c,c')\right)$]
  [Devuelve si la coordenada ingresada es valida para agregar un pokemon en ella]

  \InterfazFuncion{hayPokemonCercano}{\In{j}{juego}, \In{c}{coor}}{Bool}
  [True]
  {$res \igobs hayPokemonCercano(c,j)$}
  [$\Theta(1)$] 
  [Devuelve si la coordenada ingresada pertenece al rango de un pokemon salvaje]   

  \InterfazFuncion{posPokemonCercano}{\In{j}{juego}, \In{c}{coor}}{coor}
  [$hayPokemonCercano(c,j)$]
  {$res \igobs posPokemonCercano(c,j)$}
  [$\Theta(1)$] %Todas estas ultimas deberian ser O(1) gracias al magico DiccAcHashMagicTableListTupl pero por las dudas dejemo' la incognita hasta que sea oficial
  [Devuelve la coordenada mas del pokemon salvaje del rango siempre y cuando haya uno]   

  \InterfazFuncion{entrenadoresPosibles}{\In{c}{coor}, \In{es}{conjLineal(itConj(Jugador))}, \In{j}{juego}}{itConj(Jugador)}
  [$hayPokemonCercano(c,j) \yluego pokemonEnPos(posPokemonCercano(c,j),j).jugadoresEnRango \subseteq jugadoresConectados(c,j)$]
  {$res \igobs entrenadoresPosibles(c,pokemonEnPos(posPokemonCercano(c,j),j).jugadoresEnRango,j)$}
  [$O(Cardinal(es))$] 
  [Devuelve un iterador a los jugadores que estan esperando para atrapar al pokemon mas cercano a la coordenada ingresada]   

  \InterfazFuncion{indiceRareza}{\In{j}{juego}, \In{p}{Pokemon}}{Nat}
  [$p \in todosLosPokemons(j)$]
  {$res \igobs indiceRareza(p,j)$}
  [$O(|P|)$] 
  [Devuelve el indice de rareza del pokemon del juego ingresado]

  \InterfazFuncion{cantPokemonesTotales}{\In{j}{juego}}{Nat}
  [true]
  {$res \igobs cantPokemonsTotales(p)$}
  [$\Theta(1)$] 
  [Devuelve la cantidad de pokemones que hay en el juego]

  \InterfazFuncion{cantMismaEspecie}{\In{j}{Juego}, \In{p}{Pokemon}}{Nat}
  [true]
  {$res \igobs cantMismaEspecie(p, pokemons(j), j$}
  [$O(|P|)$] 
  [Devuelve la cantidad de pokemones de la especie ingresada hay en el juego]

  
\end{Interfaz}

\begin{Representacion}
\subsubsection{Representación de Juego}

	\begin{Estructura}{Jugador}[Nat]
	\end{Estructura}
	
	\begin{Estructura}{Pokemon}[String]
	\end{Estructura}
	
	\begin{Estructura}{Juego}[estr]
		\begin{Tupla}[estr]
			\tupItem{pokemones}{diccString(Pokemon, Nat)}%
			\tupItem{todosLosPokemones}{conjLineal(infoPokemon)}
		%	\tupItem{pokedex}{diccString(Pokemon, Nat)}
			\tupItem{idsJugadores}{conjLineal(Jugador)}
			\tupItem{jugadores}{conjLineal(infoJugador)}
			\tupItem{expulsados}{conjLineal(Jugador)}
			\tupItem{jugadoresPorID}{Vector($<$info: itConj(infoJugador), encolado: itColaPrior(Jugador)$>$)} %% itConj es un iterador a jugadores, itColaPrior es un iterador al heap en el que esta el jugador
			\tupItem{posicionesPokemons}{DiccMat(coor, itConj(infoPokemon))} %% itConj es un iterador al conjunto de todosLosPokemones
			\tupItem{posicionesJugadores}{DiccMat(coor, puntero(conjLineal(Jugador)))} %% cada coordenada tiene un conjunto de jugadores. con el jugador se lo puede busacr en jugadoresPorID, donde la primer coordenada se agrega al heap del pokemon, lo que devuelve el agregar al heap es el segundo valor en la tupla
			\tupItem{mapa}{map}
		\end{Tupla}
	\end{Estructura}
	
		\begin{Tupla}[infoJugador]
			\tupItem{id}{itConj(Jugador)}
			\tupItem{estaConectado}{Bool}
			\tupItem{sanciones}{Nat}
			\tupItem{pokemonesCapturados}{conjLineal(itConj(infoPokemon))} %% Revisar si es infoPokemon o pokemones. tiene que apuntar al conjunto de todos los pokemones
      		\tupItem{posicion}{coor}
      		\tupItem{itPosicion}{itConj(Jugador)}
		\end{Tupla}
	
		\begin{Tupla}[infoPokemon]
			\tupItem{tipo}{Pokemon}
			\tupItem{posicion}{coor}
			\tupItem{contador}{Nat}
			\tupItem{jugadoresEnRango}{colaPrior$<$Nat, itConj(Jugador)$>$}
			\tupItem{salvaje}{Bool}
		\end{Tupla}
	

\subsubsection{Invariante de Representación}
	\begin{enumerate}
		\item La suma de todas las cantidades de cada tipo de pokemon es igual al cardinal del conjunto todosLosPokemones.
		\item Para cada infoPokemon definido en todosLosPokemones, si es salvaje, su tipo esta definido en pokemones, su coordenada esta definida en posicionesPokemones y el significado en posicionesPokemones es igual a infoPokemon, su contador es un Nat menor estricto que 10 y cada jugador en rango esta en jugadores.
		\item El cardinal de idsJugadores es igual a la longitud de jugadoresPorID, que es igual al cardinal de jugadores + el cardinal de expulsados
		\item Cada jugador en idsJugadores pertenece a expulsados, o su numero de id representa la posicion en jugadoresPorID donde esta guardada su informacion. Esta informacion refleja lo definido en jugadores. Si el jugador esta encolado, entonces existe un infoPokemon en todos los pokemones, tal que ese jugador pertenece a los jugadoresEnRango del pokemon.
		\item Cada coordenada valida en el mapa, si esta definida en posicionesPokemones, su significado pertenece a todosLosPokemones y es salvaje		
		\item Cada coordenada valida en el mapa, si esta definida en posicionesJugadores, su significado esta contenido en jugadores
		\item Para cada infoJugador en jugadores, su id pertenece a idsJugadores. Si esta conectado, su itPosicion esta contenido en el significado de su posicion en posicionesJugadores. Sus sanciones son menores a 5. Sus pokemones capturados estan contenidos en todosLosPokemones
		\item Los conjuntos jugadores y expulsados son disjuntos.
	\end{enumerate}
	
	\begin{enumerate}
	\item $\left(\displaystyle\sum_{c \in claves(e.pokemones)}Obtener(e.pokemones,c)\right = \#e.todosLosPokemons)$
	\item ($\forall$ i : infoPokemon) pertenece?(e.todosLosPokemones, i) $\impluego$ (definido?(e.pokemones, i.tipo) $\land$ definido?(e.posicionesPokemones, i.posicion) $\yluego$ Significado(e.posicionesPokemones, i.posicion) = i $\land$ i.contador $<$ 10 $\land$ ($\forall$ j : Jugador) j $\in$ i.jugadoresEnRango $\impluego$ j $\in$ e.jugadores )
	\item $\#$e.idsJugadores = long(e.jugadoresPorID) $\land$ $\#$e.idsJugadores = ($\#$e.jugadores + $\#$e.expulsados)
	\item ($\forall$ j : jugador) j $\in$ e.idsJugadores $\impluego$ (j $\in$ e.expulsados $\oluego$ e.jugadoresPorID[j].info $\in$ e.jugadores $\land$ Siguiente(e.jugadoresPorID[j].encolado) $\neq$ NULL $\impluego$ (($\forall$ i : infoPokemon) i $\in$ e.todosLosPokemones $\yluego$ j $\in$ i.jugadoresEnRango))
	\item ($\forall$ c : coor) c $\in$ e.mapa.coordenadas $\impluego$ (definido?(e.posicionesPokemones, c) $\impluego$ Siguiente(Significado(e.posicionesPokemones, c)) $\in$ e.todosLosPokemones $\land$ Siguiente(Significado(e.posicionesPokemones, c))).esSalvaje
	\item ($\forall$ c : coor) c $\in$ e.mapa.coordenadas $\impluego$ (definido?(e.posicionesJugadores, c) $\impluego$ *(Significado(e.posicionesPokemones, c)) $\subseteq$ e.jugadores)
	\item ($\forall$ i : infoJugador) i $\in$ e.jugadores $\impluego$ (i.id $\in$ e.idsJugadores $\land$ i.estaConectado $\impluego$ (i.itPosicion $\in$ *Significado(e.posicionesJugadores, i.posicion)) $\land$ i.sanciones $<$ 5 $\land$ i.pokemonesCapturados $\subseteq$ e.todosLosPokemones)
	\item e.jugadores $\cap$ e.expulsados = $\emptyset$
\end{enumerate}

\Rep[estr][e]{1 $\yluego$ 2 $\yluego$ 3 $\yluego$ 4 $\yluego$ 5 $\yluego$ 6 $\yluego$ 7 $\yluego$ 8}
	
\subsubsection{Funci\'on de Abstracci\'on}	
	Abs(e): estre - $>$ Jugo {Rep(e)} 
 pGo: Juego tq e.mapa = mapa(pGo) y e.jugadores = jugadores(pGo) yluego \\
 (Para todo j : jugador) j pertenece e.jugadores impluego 
 \\ j.sanciones = sanciones(j, pGo) ((j pertenece expulsados(pGo) y j.sanciones >= 10)\\
 oluego (j.pokesCapturados = pokemones(j,pGo) y j.estaConectado = estaConectad(j,pGo) \\
 y j.estaConectado impluego j.pos = posicion(j,pGo))) y \\
 (Para todo p : pokemon) p pertenece c.pokemones impluego (Para todo j : Jugador) \\
 j pertenece e.jugadores yluego p pertenece pokemones(j,pGo) o [(Para todo c : coord)\\
 c pertenece e.mapa.coordenadas yluego p = pokemonEnPos(c,pGo) y cantMovParaCap(c,pGo)\\
 p.contador]
\end{Representacion}

%%% Sanciones	O(1)
%%% Contador	O(1)
%%% Heap		O(log(EC))
%%% Atrapar		O(PS + PC)
%%% BorrarJ		O(|P| * PC)

\subsection{Algoritmos}

\begin{algorithm}[H]{\textbf{iCrearJuego}(\In{m}{map}) $\to$ res : estr}
	\begin{algorithmic}
		\State estr : j	\Comment O(1)
		\State j.pokemones $\gets$ CrearDiccionario()	\Comment O(1)
		\State j.todosLosPokemones $\gets$ Vacio()	\Comment O(1)
%		\State j.pokedex $\gets$ CrearDiccionario()	\Comment O(1)		
		\State j.idsJugadores $\gets$ Vacio()	\Comment O(1)		
		\State j.jugadores $\gets$ Vacio()	\Comment O(1)
		\State j.expulsados $\gets$ Vacio()	\Comment O(1)
		\State j.jugadoresPorID $\gets$ Vacia()	\Comment O(1)
		\State j.posicionesPokemons $\gets$	Vacio(Largo(m), Ancho(m))	\Comment O(largo(m) x ancho(m))
		\State j.posicionesJugadores $\gets$	Vacio(Largo(m), Ancho(m))	\Comment O(largo(m) x ancho(m))
		\State j.mapa $\gets$ m	\Comment $O(copy(m))$

		\State $res \gets j$ \Comment O(1)
		\medskip
		\Statex \underline{Complejidad:} O((largo(m) x ancho(m)) + copy(m))
		\Statex \underline{Justificación:} 

    \end{algorithmic}
\end{algorithm}

\begin{algorithm}[H]{\textbf{iAgregarPokemon}(\Inout{j}{estr}, \In{c}{coor}, \In{p}{Pokemon}) $\to$ res : itConj(Pokemon)}
	\begin{algorithmic}
		\State infoP $\gets$ $<$p, c, 0, Vacio(), True$>$	\Comment O(|P|)
		\State itPokemon $\gets$ AgregarRapido(j.todosLosPokemones, infoP)	\Comment O(copy(infoP)) {Copiar los elementos de la tupla es O(1)}
		\State desdeLat $\gets$ iDamePos(iLatitud(c), 2) \Comment O(1)
		\State desdeLong $\gets$ iDamePos(iLongitud(c), 2) \Comment O(1)
		\While {desdeLat $\leq$ Latitud(c) + 2}	\Comment O(1) {Recorro un conjunto acotado de coordenadas}
			\While {desdeLong $\leq$ Longitud(c) + 2}	\Comment O(1) {Recorro un conjunto acotado de coordenadas}
				\If {DistEuclidea(CrearCoor(desdeLat, desdeLong), c) $\leq$ 4} \Comment O(1)
					\If {Definido?(j.posicionesJugadores, CrearCoor(desdeLat, desdeLong))}	\Comment O(1)
						\State itJugadores $\gets$ iCrearIt(*(Significado(j.posicionesJugadores, CrearCoor(desdeLat, desdeLong))))	\Comment O(1)
						\While {HaySiguiente(itJugadores)}	\Comment O(EC)
							\State e $\gets$ Siguiente(itJugadores)	\Comment O(1)
							\State tupJugId $\gets$ j.jugadoresPorID[e]	\Comment O(1)
							\State cantPokemonesJug $\gets$ Cardinal(Siguiente(tupJugId$_{1}$).pokemonesCapturados)	\Comment O(1)
							\State itCola $\gets$ iEncolar(Siguiente(itPokemon).jugadoresEnRango, cantPokemonesJug, e)	\Comment O(log(EC) + copy(e))
							\State tupJugId$_{2}$ $\gets$ itCola	\Comment O(1)
							\State Avanzar(itJugadores)	\Comment O(1)
						\EndWhile
					\EndIf
				\EndIf
			\EndWhile
		\EndWhile
		\If {Definido?(j.pokemones, p)}	\Comment O(|P|)
			\State cViejo $\gets$ Obtener(j.pokemones, p)	\Comment O(|P|)
			\State Eliminar(j.pokemones, p)	\Comment O(|P|)
			\State Definir(j.pokemones, p, cViejo + 1)	\Comment O(|P|)
		\Else
			\State Definir(j.pokemones, p, 1)	\Comment O(|P|)
		\EndIf
		\State res $\gets$ itPokemon	\Comment O(1)
	
		\medskip
		\Statex \underline{Complejidad:} $O(|P| + |EC| * log(|EC|))$
		\Statex \underline{Justificación:}
    \end{algorithmic}
\end{algorithm}

\begin{algorithm}[H]{\textbf{iAgregarJugador}(\Inout{j}{estr}) $\to$ res : Jugador}
	\begin{algorithmic}
		\State id $\gets$ Cardinal(j.jugadores) + Cardinal(j.expulsados) \Comment O(1)
		\State itConjIds $\gets$ AgregarRapido(j.idsJugadores, id)	\Comment O(1)
		\State infoJ $\gets$ $<$itConjIds, false, 0, Vacio(), CrearCoor(0, 0), CrearIt(Vacio())$>$ \Comment O(1) {El jugador es creado vacio, sin sanciones y sin pokemones atrapados. La coordenada es una cualquiera, ya que solo se va a acceder a ella cuando el jugador este conectado, y al conectarse, se le asigna una valida}
		\State itJ $\gets$ AgregarRapido(j.jugadores, infoJ) \Comment O(copy(infoJ)) {Se puede utilizar AgregarRapido porque infoJ contiene un iterador al id del jugador, que es univoco}
		\State AgregarAtras(j.jugadoresPorID, <itJ, NULL>) \Comment O(J) {Donde J es la cantidad total de jugadores que fueron agregados al juego} %% O(J)??? revisar!
		\State res $\gets$ id \Comment O(1)
					
		\medskip
		\Statex \underline{Complejidad:} $O(J)$ donde J es la cantidad de jugadores
		\Statex \underline{Justificación:} O(copy(Jugador)) es igual a O(1) ya que solamente es copiar Nat, Bool y un conjunto vacio.
    \end{algorithmic}
\end{algorithm}

\begin{algorithm}[H]{\textbf{iConectarse}(\Inout{j}{estr}, \In{e}{Jugador}, \In{c}{coor})}
	\begin{algorithmic}
		\State itJug $\gets$ $j.jugadoresPorId[e]_{1}$	\Comment O(1)
    \State infoJ $\gets$ Siguiente(itJug) \Comment O(1) 
		\State infoJ.estaConectado $\gets$ true \Comment O(1)
    \State AgregarJugadorEnPos(j.posicionesJugadores, infoJ, c) \Comment O(1)
    \State infoJ.estaConectado $\gets$ true \Comment O(1)
		\If {HayPokemonCercano(j, c)}	\Comment O(1)
			\State p $\gets$ Siguiente(Significado(j.posicionesPokemons, PosPokemonCercano(j, c)))	\Comment O(1)
			\State itJug $\gets$ Encolar(p.jugadoresEnRango, Cardinal(jug.pokeCapturados), itJug)	\Comment O(log(EC))
			\State p.contador $\gets$ 0	\Comment O(1)
		\EndIf
	
		\medskip
		\Statex \underline{Complejidad:} O(log(EC))
		\Statex \underline{Justificación:} EC es la maxima cantidad de jugadores esperando para atrapar un pokemon. En el peor caso, el heap al que entra el jugador es el que mas jugadores esperando tiene.
    \end{algorithmic}
\end{algorithm}


\begin{algorithm}[H]{\textbf{iAgregarJugadorEnPos}(\Inout{d}{DiccMat(coor, puntero(conjLineal(Jugador)))}, \Inout{j}{infoJugador}, \In{c}{coor})} 

$\textbf{Pre}$ $\equiv$ $\{$ d $=$ d$_0$ $\land$ longitud(c) $<$ ancho(d) $\land$ latitud(c) $<$ largo(d) $\}$
  \begin{algorithmic}
      \State conLineal(Jugador) : jugsEnPos \Comment O(1)
      \If {$\lnot$ Definido?(d, c)}	\Comment O(1)
          \State Definir(d, c, $\&Vacio()$) \Comment O(1)
      \EndIf
      \State jugsEnPos $\gets$ *(Significado(d, c)) \Comment O(1)
      \State j.itPosicion $\gets$ AgregarRapido(jugsEnPos, j.id) \Comment O(1)
      \State j.posicion $\gets$ c \Comment O(1)
		\Statex \underline{Complejidad:} O(1) 
    \end{algorithmic}
$\textbf{Post}$ $\equiv$ $\{$d = definir(logitud(c), latitud(c), Ag(j, significado(longitud(c), latitud(c))))$\}$
\end{algorithm}

\begin{algorithm}[H]{\textbf{iDesconectarse}(\Inout{j}{estr}, \In{id}{Jugador})} 
	\begin{algorithmic}
		\State tupJug $\gets$ j.jugadoresPorId[e]	\Comment O(1)
		\If {HayPokemonCercano(j, c)}	\Comment O(1)
			\State $tupJug_{2}$ $\gets$ EliminarSiguiente($tupJug_{2}$)	\Comment $O(log(EC))$
		\EndIf
    \State infoJ $\gets$ Siguiente($tupJug_{1}$) \Comment O(1) 
		\State infoJ.estaConectado $\gets$ false	\Comment O(1)
    \State EliminarSiguiente(infoJ.itPosicion) \Comment O(1)
    \State infoJ.itPosicion $\gets$ CrearIt(Vacio())	\Comment O(1)
		\medskip
		\Statex \underline{Complejidad:} $O(log(EC))$
		\Statex \underline{Justificación:} EC es la maxima cantidad de jugadores esperando para atrapar un pokemon. En el peor caso, el heap del que sale el jugador es el que mas jugadores esperando tiene.
    \end{algorithmic}
\end{algorithm}

\begin{algorithm}[H]{\textbf{iMoverse}(\Inout{j}{estr}, \In{id}{Jugador}, \In{c}{coor})}{}
	\begin{algorithmic}
	\State tupJug $\gets$ j.jugadoresPorID[id]	\Comment O(1)
	\State infoJ $\gets$ Siguiente(tupJug$_{1}$)	\Comment O(1)
	\If {iHayPokemonCercano(c, j)}	\Comment O(1) {Pregunto si me estoy moviendo a donde hay un pokemon}
		\State p $\gets$ iPosPokemonCercano(j, c)	\Comment O(1)
		\If{$\neg$ iHayPokemonCercano(infoJ.posicion, j)}	\Comment O(1) {Pregunto si no estaba en rango de un pokemon antes de moverme}
			\State infoP $\gets$ Siguiente(Significado(j.posicionesPokemons, p))	\Comment O(1)
			\State infoP.contador $\gets$ 0 \Comment O(1)
			\State cantPokemonesJug $\gets$ Cardinal(infoJ.pokemonesCapturados)	\Comment O(1)
			\State tupJugId$_{2}$ $\gets$ iEncolar(infoP.jugadoresEnRango, cantPokemonesJug, id)	\Comment O(log(EC))
		\Else
			\If {p $\neq$ iPosPokemonCercano(j, infoJ.posicion)}	\Comment O(1) {Pregunto si el pokemon donde estaba es distinto del que voy a estar}
				\State infoP $\gets$ Siguiente(Significado(j.posicionesPokemons, p))	\Comment O(1)
			\State infoP.contador $\gets$ 0 \Comment O(1)
			\State cantPokemonesJug $\gets$ Cardinal(infoJ.pokemonesCapturados)	\Comment O(1)
			\State EliminarSiguiente(tupJugId$_{2}$)	\Comment O(log(EC))
			\State tupJugId$_{2}$ $\gets$ iEncolar(infoP.jugadoresEnRango, cantPokemonesJug, id)	\Comment O(log(EC))
			\EndIf
		\EndIf
		\State iActualizarMenos(p, j)	\Comment O(PS)
	\Else
		\If {iHayPokemonCercano(infoJ.posicion, j)}	\Comment O(1) {Ya se que no hay pokemon donde voy a estar, si hay pokemon donde estoy ahora es que salgo de un rango}
			\State EliminarSiguiente(tupJugId$_{2}$)	\Comment O(log(EC)) {Elimina al jugador del heap donde estaba esperando}
		\EndIf
		\State iActualizarTodos(j)	\Comment O(PS)

	\EndIf
	\If {debeSancionarse?(id, c, j)}
		\If {infoJ.sanciones $<$ 4}	\Comment O(1)
			\State infoJ.sanciones $\gets$ infoJ.sanciones + 1	\Comment O(1)
			\State EliminarSiguiente(infoJ.itPosicion) \Comment O(1)
			\State AgregarJugadorEnPos(j.posicionesJugadores, infoJ, c) \Comment O(1)
		\Else
			\State itPCapturados $\gets$ CrearIt(infoJ.pokemonesCapturados)	\Comment O(1)
			\While {HaySiguiente(itPCapturados)}	\Comment O(PC)
				\State infoP $\gets$ Siguiente(Siguiente(itPCapturados))	\Comment O(1)
				\State pViejos $\gets$ Obtener(j.pokemones, infoP.tipo)	\Comment O(|P|)
				\State Borrar(j.pokemones, infoP.tipo)	\Comment O(|P|)
				\If {pViejos $\neq$ 1}	\Comment O(1)
					\State Definir(j.pokemones, infoP.tipo, pViejos - 1)	\Comment O(|P|)
				\EndIf
				\State EliminarSiguiente(Siguiente(itPCapturados))	\Comment O(1)		
			\EndWhile
			\State iEliminarSiguiente(tupJug$_{1}$)	\Comment O(1) {Saca al jugador del conjunto de jugadores}
			\State AgregarRapido(j.expulsados, id)	\Comment O(copy(id)) {Copiar un nat es O(1)}
			\If {iHayPokemonCercano(j, c)}	\Comment O(1)
				\State iEliminarSiguiente(tupJug$_{2}$)	\Comment O(log(EC)) {Elimina al jugador del heap donde estaba esperando}
			\EndIf
		\EndIf
	\Else
		\State EliminarSiguiente(infoJ.itPosicion) \Comment O(1)
		\State AgregarJugadorEnPos(j.posicionesJugadores, infoJ, c) \Comment O(1)
	\EndIf

\medskip
\Statex \underline{Complejidad:} O(PS + (PC * |P|) + log(EC))
\Statex \underline{Justificación:}
\end{algorithmic}
\end{algorithm}

\begin{algorithm}[H]{\textbf{iActualizarMenos}(\Inout{j}{estr}, \In{c}{coor})} 
	\begin{algorithmic}
    \State p $\gets$ iPosPokemonCercano(j, c)	\Comment O(1)
		\State itCoor $\gets$ Coordenadas(j.posicionesPokemons)	\Comment O(1)
		\While {HaySiguiente(itCoor)}	\Comment O(PS)
			\If{Siguiente(itCoor) $\neq$ p}	\Comment O(1)
				\State itPokemon $\gets$ Significado(j.posicionesPokemons, Siguiente(itCoor))	\Comment O(1)
					\State ActualizarPokemon(itPokemon)	\Comment O(1)
			\EndIf
		\EndWhile

	
		\medskip
		\Statex \underline{Complejidad:} O(PS)
		\Statex \underline{Justificación:} Actualiza todos las colas de prioridad excepto por la que este en la posicion que paso por parametro.
    \end{algorithmic}
\end{algorithm}

\begin{algorithm}[H]{\textbf{iActualizarTodos}(\Inout{j}{estr})} 
	\begin{algorithmic}
		\State itCoor $\gets$ Coordenadas(j.posicionesPokemons)	\Comment O(1)
		\While {HaySiguiente(itCoor)}	\Comment O(PS)
			\State itPokemon $\gets$ Significado(j.posicionesPokemons, Siguiente(itCoor))	\Comment O(1)
				\State ActualizarPokemon(itPokemon)	\Comment O(1)
		\EndWhile

	
		\medskip
		\Statex \underline{Complejidad:} O(PS)
		\Statex \underline{Justificación:} Actualiza todos las colas de prioridad. invalido.
    \end{algorithmic}
\end{algorithm}

\begin{algorithm}[H]{\textbf{iActualizarPokemon}(\Inout{itPokemones}{itConj(infoPokemon)}, \Inout{j}{estr})} 
	\begin{algorithmic}
		\State infoP $\gets$ Siguiente(itPokemones)	\Comment O(1)
		\State infoP.contador $\gets$ infoP.contador + 1	\Comment O(1)
    \If {infoP.contador = 10}	\Comment O(1)
			\State e $\gets$ Proximo(infoP.jugadoresEnRango)	\Comment O(1)
			\State infoP.esSalvaje $\gets$ false	\Comment O(1)
			\State infoJ $\gets$ Siguiente(j.jugadoresPorID[e]$_{1}$)	\Comment O(1)
			\State AgregarRapido(infoJ.pokemonesCapturados, itPokemones)	\Comment O(copy(itPokemones)) {Copiar el iterador es O(1)}
			\State Borrar(j.posicionesPokemons, infoP.posicion) \Comment O(1)
		\EndIf

	
		\medskip
		\Statex \underline{Complejidad:} O(1)
		\Statex \underline{Justificación:} Actualiza la cola de prioridad del pokemon que paso por parametro. Suma uno al contador en caso de que no tenga que atraparse, si tiene que ser atrapado, lo agrega al conjunto de pokemones atrapados del jugador y lo elimina del diccionario de pokemones por posicion.
    \end{algorithmic}
\end{algorithm}

\begin{algorithm}[H]{\textbf{iDebeSancionarse}(\In{e}{Jugador}, \In{c}{coor}, \In{j}{estr}) $\to$ res : Bool} 
	\begin{algorithmic}
    \State tupJug $\gets$ j.jugadoresPorID[e]	\Comment O(1)
  	\State infoJ $\gets$ Siguiente(tupJug$_{1}$)	\Comment O(1)
		\State $res \gets \neg HayCamino(infoJ.posicion, c, j.mapa) \lor DistEuclidea(infoJ.posicion, c, j.mapa) > 100 $ \Comment O(1)
	
		\medskip
		\Statex \underline{Complejidad:} O(1)
		\Statex \underline{Justificación:} Checkea si el jugador hizo un movimiento invalido.
    \end{algorithmic}
\end{algorithm}

\begin{algorithm}[H]{\textbf{iMapa}(\In{j}{estr}) $\to$ res : map} 
	{}
	\begin{algorithmic}
		\State $res \gets j.mapa $ \Comment O(copy(mapa(j)))
	
		\medskip
		\Statex \underline{Complejidad:} O(copy(mapa(j)))
		\Statex \underline{Justificación:} Devuelve el mapa del juego por copia.
    \end{algorithmic}
\end{algorithm}

\begin{algorithm}[H]{\textbf{iJugadores}(\In{j}{estr}) $\to$ res : itConj(Jugador)} 
	{}
	\begin{algorithmic}
		\State res $\gets$ CrearIt(j.idsJugadores)	\Comment O(1)
	
		\medskip
		\Statex \underline{Complejidad:} O(1)
		\Statex \underline{Justificación:} Devuelve el mapa del juego.
    \end{algorithmic}
\end{algorithm}

\begin{algorithm}[H]{\textbf{iEstaConectado}(\In{j}{estr}, \In{id}{Jugador}) $\to$ res : Bool} 
	{}
	\begin{algorithmic}
		\State $res \gets Siguiente(j.jugadoresPorID[id]_0).estaConectado $ \Comment O(1)
	
		\medskip
		\Statex \underline{Complejidad:} O(1)
		\Statex \underline{Justificación:} Devuelve si el jugador esta conectado.
    \end{algorithmic}
\end{algorithm}

\begin{algorithm}[H]{\textbf{iPosicion}(\In{j}{estr}, \In{id}{Jugador}) $\to$ res : coor} 
	{}
	\begin{algorithmic}
		\State res $\gets$ Siguiente((j.jugadoresPorID[id]).info).posicion  \Comment O(1)
	
		\medskip
		\Statex \underline{Complejidad:} O(1)
		\Statex \underline{Justificación:} Devuelve la posicion del jugador.
    \end{algorithmic}
\end{algorithm}

\begin{algorithm}[H]{\textbf{iPokemones}(\In{j}{estr}, \In{id}{Jugador}) $\to$ res : itConj($<$Pokemon, Nat$>$)} 
	{}
	\begin{algorithmic}
		\State $res \gets CrearIt(Siguiente((j.jugadoresPorID[id]_1).pokeCapturados) $ \Comment O(1)
	
		\medskip
		\Statex \underline{Complejidad:} O(1)
		\Statex \underline{Justificación:} Devuelve un iterador al conjunto de pokemones atrapados por el jugador.
    \end{algorithmic}
\end{algorithm}

\begin{algorithm}[H]{\textbf{iExpulsados}(\In{j}{estr}) $\to$ res : itConj(Jugador)} 
	{}
	\begin{algorithmic}
		\State $res \gets CrearIt(j.expulsados) $ \Comment O(1)
	
		\medskip
		\Statex \underline{Complejidad:} O(1)
		\Statex \underline{Justificación:} Devuelve un iterador al conjunto de jugadores expulsados.
    \end{algorithmic}
\end{algorithm}

\begin{algorithm}[H]{\textbf{iPosConPokemones}(\In{j}{estr}) $\to$ res : itConj(coor)} 
	{}
	\begin{algorithmic}
		\State $res \gets CrearIt(Coordenadas(j.posicionesPokemons)) $ \Comment O(1)
	
		\medskip
		\Statex \underline{Complejidad:} O(1)
		\Statex \underline{Justificación:} Devuelve las posiciones con pokemones.
    \end{algorithmic}
\end{algorithm}

\begin{algorithm}[H]{\textbf{iPokemonEnPos}(\In{j}{estr}, \In{c}{coor}) $\to$ res : Pokemon} 
	{}
	\begin{algorithmic}
		\State $res \gets Significado(j.posicionesPokemons, c)_{1} $ \Comment O(1)
	
		\medskip
		\Statex \underline{Complejidad:} O(1)
		\Statex \underline{Justificación:} Devuelve el mapa del juego.
    \end{algorithmic}
\end{algorithm}

\begin{algorithm}[H]{\textbf{iCantMovimientosParaCaptura}(\In{j}{estr}, \In{c}{coor}) $\to$ res : Nat} 
	{}
	\begin{algorithmic}
		\State res $\gets$ 10 - Siguiente(Significado(j.posicionesPokemons, c)).contador  \Comment O(1)
	
		\medskip
		\Statex \underline{Complejidad:} O(1)
		\Statex \underline{Justificación:} Devuelve cuantos movimientos faltan para capturar al pokemon.
    \end{algorithmic}
\end{algorithm}

\begin{algorithm}[H]{\textbf{iPosPokemonCercano}(\In{j}{estr}, \In{c}{coor})  $\to$ res : coor} 
	{}
	\begin{algorithmic}
		\State i $\gets$ 0 \Comment O(1)
		\State latC $\gets$ Latitud(c)	\Comment O(1)
		\State i $\gets$ DamePos(latC, 2) \Comment O(1)		
		\State longC $\gets$ Longitud(c)	\Comment O(1)
		\State j $\gets$ DamePos(longC, 2) \Comment O(1)
		\While {i $\leq$ latC + 2}	\Comment O(1) {Vale porque estoy recorriendo un conjunto acotado de coordenadas}
			\While {j $\leq$ longC + 2}	\Comment O(1) {Vale porque estoy recorriendo un conjunto acotado de coordenadas}
				\If {Definido?(j.posicionesPokemons, $<$i, j$>$) $\land$ DistEuclidea(c, $<$i, j$>$) $\leq$ 4}	\Comment O(1)
					\State res $\gets$ $<$i, j$>$	\Comment O(1) 
				\EndIf
				\State j $\gets$ j + 1	\Comment O(1)
			\EndWhile
			\State i $\gets$ i + 1	\Comment O(1)
		\EndWhile
		\medskip
		\Statex \underline{Complejidad:} O(1)
		\Statex \underline{Justificación:} Como el rango a recorrer es una constante, se puede decir que es de la clase $\Theta(1)$ 
    \end{algorithmic}
\end{algorithm}

\begin{algorithm}[H]{\textbf{iPuedoAgregarPokemon}(\In{j}{estr}, \In{c}{coor}) $\to$ res : Bool} 
	\begin{algorithmic}
		\State res $\gets$ PosExistente(c, j.mapa) $\land$ $\lnot$(Definido?(j.posicionesPokemons, c)) $\land$ $\lnot$(HayPokemonEnTerritorio(j, c))  \Comment $\Theta\left(\displaystyle\sum_{c' \in coordendas(mapa(j))}equal(c,c')\right)$
	
		\medskip
		\Statex \underline{Complejidad:} [$\Theta\left(\displaystyle\sum_{c' \in coordendas(mapa(j))}equal(c,c')\right)$]
		\Statex \underline{Justificación:} Tiene que ver si la posicion existe en el mapa, las demas operaciones son O(1)
     \end{algorithmic}
 \end{algorithm}
 
 \begin{algorithm}[H]{\textbf{iHayPokemonCercano}(\In{j}{estr}, \In{c}{coor}) $\to$ res : Bool} 
	\begin{algorithmic}
		\State res $\gets$ false	\Comment O(1) 
		\State latC $\gets$ Latitud(c)	\Comment O(1)
		\State i $\gets$ DamePos(latC, 2) \Comment O(1)		
		\State longC $\gets$ Longitud(c)	\Comment O(1)
		\State j $\gets$ DamePos(longC, 2) \Comment O(1)
		\While {i $\leq$ latC + 2}	\Comment O(1) {Vale porque estoy recorriendo un conjunto acotado de coordenadas}
			\While {j $\leq$ longC + 2 }	\Comment O(1) {Vale porque estoy recorriendo un conjunto acotado de coordenadas}
				\If {Definido?(j.posicionesPokemons, $<$i, j$>$) $\land$ DistEuclidea(c, $<$i, j$>$) $\leq$ 4}	\Comment O(1)
					\State res $\gets$ true	\Comment O(1) 
				\EndIf
				\State j $\gets$ j + 1	\Comment O(1)
			\EndWhile
			\State i $\gets$ i + 1	\Comment O(1)
		\EndWhile
	
		\medskip
		\Statex \underline{Complejidad:} $\Theta(1)$
		\Statex \underline{Justificación:} Como el rango a recorrer es una constante, se puede decir que es de la clase $\Theta(1)$
     \end{algorithmic}
 \end{algorithm}
 
 \begin{algorithm}[H]{\textbf{iHayPokemonEnTerritorio}(\In{j}{estr}, \In{c}{coor}) $\to$ res : Bool} 
	\begin{algorithmic}
		\State res $\gets$ false	\Comment O(1) 
		\State latC $\gets$ Latitud(c)	\Comment O(1)
		\State i $\gets$ DamePos(latC, 5) \Comment O(1)		
		\State longC $\gets$ Longitud(c)	\Comment O(1)
		\State j $\gets$ DamePos(longC, 5) \Comment O(1)
		\While {i $\leq$ latC + 5}	\Comment O(1) {Vale porque estoy recorriendo un conjunto acotado de coordenadas}
			\While {j $\leq$ longC + 5 }	\Comment O(1) {Vale porque estoy recorriendo un conjunto acotado de coordenadas}
				\If {Definido?(j.posicionesPokemons, $<$i, j$>$) $\land$ DistEuclidea(c, $<$i, j$>$) $\leq$ 25}	\Comment O(1)
					\State res $\gets$ true	\Comment O(1) 
				\EndIf
				\State j $\gets$ j + 1	\Comment O(1)
			\EndWhile
			\State i $\gets$ i + 1	\Comment O(1)
		\EndWhile
	
		\medskip
		\Statex \underline{Complejidad:} $\Theta(1)$
		\Statex \underline{Justificación:} Como el rango a recorrer es una constante, se puede decir que es de la clase $\Theta(1)$
     \end{algorithmic}
 \end{algorithm}

\begin{algorithm}[H]{\textbf{iEntrenadoresPosibles}(\In{c}{coor}, \In{es}{conjLineal(Jugador)}, \In{j}{estr})  $\to$ res : conjLineal(itConj(Jugador))} 
	{}
	\begin{algorithmic}
		\State ePosibles $\gets$ Vacia()  \Comment O(1) {Crea un conjunto de iteradores vacio}
		\If {Cardinal(es) $!=$ 0}	\Comment O(1)
			\State itE $\gets$ CrearIt(es)	\Comment O(1)
			\While {HaySiguiente(itE)} \Comment $O(Cardinal(es))$ {Es la cantidad de jugadores que haya en el conjunto es}
				\State e $\gets$ Siguiente(itE)	\Comment O(1)
				\State infoJ $\gets$ Siguiente(j.jugadoresPorID[e].info)	\Comment O(1)
				\If {infoJ.estaConectado}	\Comment O(1)
					\State posJugador $\gets$ infoJ.posicion	\Comment O(1)
					\If {(iHayPokemonCercano(posJugador, j)}	\Comment O(1)
					 	\If {iPosPokemonCercano(posJugador, j) = c}	\Comment O(1)
					 		\If {iHayCamino(c, posJugador, Mapa(j)))}	\Comment O(1)
								\State AgregarRapido(ePosibles, e)	\Comment O(copy(e)) {Copiar un Nat es O(1)}
							\EndIf
						\EndIf
					\EndIf
				\EndIf
				\State Avanzar(itE)	\Comment O(1)
			\EndWhile
		\EndIf
		\State $res \gets ePosibles$ \Comment O(1)
	
		\medskip
		\Statex \underline{Complejidad:} $O(\#(es))$
		\Statex \underline{Justificación:} Se itera por completo el conjunto de jugadores 'es'. En peor caso, todos los elementos de 'es' deben ser agregados al resultado.
    \end{algorithmic}
\end{algorithm}

\begin{algorithm}[H]{\textbf{iIndiceRareza}(\In{j}{estr}, \In{p}{Pokemon})  $\to$ res : Nat} 
	{}
	\begin{algorithmic}
		\State $cuantosP \gets iCantMismaEspecie(j, p) $ \Comment $O(|p.tipo|)$
		\State res $\gets$ 100 - (100 x cuantosP / iCantPokemonesTotales)  \Comment O(1)
	
		\medskip
		\Statex \underline{Complejidad:} $O(|P|)$
		\Statex \underline{Justificación:} Siendo $|P|$ el nombre mas largo para un pokemon en el juego.
    \end{algorithmic}
\end{algorithm}

\begin{algorithm}[H]{\textbf{iCantPokemonesTotales}(\In{j}{estr}) $\to$ res : Nat} 
	{}
	\begin{algorithmic}
		\State res $\gets$ Cardinal(j.todosLosPokemones)	\Comment O(1)
	
		\medskip
		\Statex \underline{Complejidad:} O(1)
		\Statex \underline{Justificación:} Pide el cardinal de un conjunto.
    \end{algorithmic}
\end{algorithm}

\begin{algorithm}[H]{\textbf{iCantMismaEspecie}(\In{j}{estr}, \In{p}{Pokemon}) $\to$ res : Nat} 
	{}
	\begin{algorithmic}
		\If {Definido?(j.pokemones, p.tipo)} \Comment $O(|P|)$
			\State res $\gets$ Longitud(Obtener(j.pokemones, p.tipo)) \Comment $O(|P|)$
		\Else
			\State res $\gets$ 0  \Comment O(1)
		\EndIf
	
		\medskip
		\Statex \underline{Complejidad:} $O(|P|)$
		\Statex \underline{Justificación:} En peor caso, el pokemon que se busca es el de nombre mas largo o no esta en el diccionario.
    \end{algorithmic}
\end{algorithm}

\begin{algorithm}[H]{\textbf{DamePos}(\In{p}{Nat}, \In{step}{Nat}) $\to$ res : Nat} 
	{}
	\begin{algorithmic}
    \If {p $\geq$ step} \Comment O(1)
      \State res $\gets$ p - step \Comment O(1)
    \Else
      \State res $\gets$ 0 \Comment O(1)
    \EndIf
		\medskip
		\Statex \underline{Complejidad:} O(1)
		\Statex \underline{Justificación:} Aritmetica de naturales.
    \end{algorithmic}
\end{algorithm}
