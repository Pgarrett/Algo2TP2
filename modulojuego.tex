\section{Modulo Juego}

%TODO descripcion

\begin{Interfaz}

  \textbf{usa}: \tadNombre{Mapa, Coordenada}.

  \textbf{se explica con}: \tadNombre{Juego}.

  \textbf{generos}: \TipoVariable{juego}.

  \InterfazFuncion{CrearJuego}{\In{m}{mapa}}{juego}%
  [true]
  {$res$ \igobs crearJuego($m_0$) $\land$ mapa($res$) \igobs $m_0$}
  [$\Theta(MUCHO)$]
  [Crea el nuevo juego, revisar la complejidad]

  \InterfazFuncion{AgregarPokemon}{\Inout{j}{juego}, \In{c}{coor}, \In{p}{pokemon}}{itPokemon}
  [$j \igobs j_0 \land puedoAgregarPokemon(c, j_0)$]
  {$j \igobs agregarPokemon(p, c, j_0)$}
  [$O(|P| + EC * log(EC))$]
  [EC es la maxima cantidad de jugadores esperando para atrapar un pokemon. |P| es el nombre mas largo para un pokemon en el juego]

  \InterfazFuncion{AgregarJugador}{\Inout{j}{juego}}{Nat}
  [$j \igobs j_0$]
  {$j \igobs agregarJugador(j_0) \land res = \#jugadores(j_0) + \#expulsados(j_0)$}
  [$O(J)$]
  [Agrega el jugador en el conjLineal, el iterador que devuelve el agregar se guarda en un vector donde la posicion es el id del jugador que voy a devolver]
  
  \InterfazFuncion{Conectarse}{\Inout{j}{juego}, \In{id}{Nat}, \In{c}{coor}}{}
  [$j \igobs j_0 \land id \in jugadores(j_0) \yluego \neg estaConectado(id, j_0) \land posExistente(c, mapa(j_0))$]
  {$j \igobs conectarse(id, c, j_0)$}
  [$O(log(EC))$]
  [Conecta al jugador pasado por parametro en la coordenada indicada]
  
  \InterfazFuncion{Desconectarse}{\Inout{j}{juego}, \In{id}{Nat}}{}
  [$j \igobs j_0 \land id \in jugadores(j_0) \yluego estaConectado(id, j_0)$]
  {$j \igobs desconectarse(id, j_0)$}
  [$O(log(EC))$]
  [Desconecta al jugador pasado por parametro]
  
  \InterfazFuncion{Moverse}{\Inout{j}{juego}, \In{id}{Nat}, \In{c}{coor}}{}
  [$j \igobs j_0 \land id \in jugadores(j_0) \yluego estaConectado(id, j_0) \land posExistente(c, mapa(j_0))$]
  {$j \igobs moverse(c, id, j_0)$}
  [$O((PS + PC) * |P| + log(EC))$]
  [Mueve al jugador pasado por parametro a la coordenada indicada]

  \InterfazFuncion{Mapa}{\In{j}{juego}}{Mapa}
  [true]
  {$res \igobs mapa(j)$}
  [$O(copy(mapa(j)))$]
  [Devuelve el mapa del juego]

  \InterfazFuncion{Jugadores}{\In{j}{juego}}{itConj(Jugador)}
  [true]
  {$res \igobs jugadores(j)$}
  [$\Theta(1)$]
  [Devuelve un iterador al conjunto de jugadores del juego]

  \InterfazFuncion{estaConectado}{\In{j}{juego}, \In{id}{Nat}}{Bool}
  [id $\in$ jugadores(j)]
  {$res \igobs estaConetado(id,j)$}
  [$\Theta(1)$]
  [Devuelve si el jugador con id ingresado esta conectado o no]

  \InterfazFuncion{posicion}{\In{j}{juego}, \In{id}{Nat}}{coor}
  [id $\in$ jugadores(j) $\yluego$ estaConectado(id,j)]
  {$res \igobs posicion(id,j)$}
  [$\Theta(1)$]
  [Devuelve la posicion actual del jugador con id ingresado si esta conectado]

  \InterfazFuncion{pokemones}{\In{j}{juego}, \In{id}{Nat}}{itLista(puntero(pokemon))} %%Para borrar todos los pokemones a la hora de eliminar al jugador tenemos complejidad lineal, pero hace falta una estructura que admita repetidos ya que lo que devuelve es un multiconjunto, no queremos hacer un modulo multiconjunto queremos ser felices. Despues debatimos
  [id $\in$ jugadores(j)]
  {$res \igobs pokemons(id,j)$}
  [$\Theta(1)$]
  [Devuelve un iterador a la estructura que almacena los punteros a pokemons del jugador del id ingresado]

  \InterfazFuncion{expulsados}{\In{j}{juego}}{itConj(Jugador)}
  [True]
  {$res \igobs expulsados(j)$}
  [$\Theta(1)$]
  [Devuelve un iterador al conjunto de jugadores expulsados del juego]

  \InterfazFuncion{posConPokemones}{\In{j}{juego}}{itConj(Coor)}
  [True]
  {$res \igobs posConPokemons(j)$}
  [$O(1)$]
  [Devuelve un iterador al conjunto de coordenadas en donde hay pokemons]

  \InterfazFuncion{pokemonEnPos}{\In{j}{juego}, \In{c}{Coor}}{itPokemon}
  [c $\in$ posConPokemons(j)]
  {$res \igobs pokemonEnPos(c,j)$}
  [$\Theta(1)$] %Pero podemos tomarlos lo que queramos
  [Devuelve un iterador al pokemon de la coordenada dada]

  \InterfazFuncion{cantMovimientosParaCaptura}{\In{j}{juego}, \In{c}{Coor}}{Nat}
  [c $\in$ posConPokemons(j)]
  {$res \igobs cantMovimientosParaCaptura(c,j)$}
  [$\Theta(1)$] %Pero podemos tomarnos lo que queramos (?) no, enrealidad no.
  [Devuelve la cantidad de movimientos acumulados hasta el momento, para atrapar al pokemon de la coordenada dada]

  \InterfazFuncion{puedoAgregarPokemon}{\In{j}{juego}, \In{c}{Coor}}{Bool}
  [True]
  {$res \igobs puedoAgregarPokemon(c,j)$}
  [$\Theta(???)$] 
  [Devuelve si la coordenada ingresada es valida para agregar un pokemon en ella]

  \InterfazFuncion{hayPokemonCercano}{\In{j}{juego}, \In{c}{Coor}}{Bool}
  [True]
  {$res \igobs hayPokemonCercano(c,j)$}
  [$\Theta(???)$] 
  [Devuelve si la coordenada ingresada pertenece al rango de un pokemon salvaje]   

  \InterfazFuncion{posPokemonCercano}{\In{j}{juego}, \In{c}{Coor}}{Coor}
  [$hayPokemonCercano(c,j)$]
  {$res \igobs posPokemonCercano(c,j)$}
  [$\Theta(???)$] %Todas estas ultimas deberian ser O(1) gracias al magico DiccAcHashMagicTableListTupl pero por las dudas dejemo' la incognita hasta que sea oficial
  [Devuelve la coordenada mas del pokemon salvaje del rango siempre y cuando haya uno]   

  \InterfazFuncion{entrenadoresPosibles}{\In{j}{juego}, \In{c}{Coor}}{itColaPrior(itJugador)}
  [$hayPokemonCercano(c,j) \yluego pokemonEnPos(posPokemonCercano(c,j),j).jugadoresEnRango \subseteq jugadoresConectados(c,j)$]
  {$res \igobs entrenadoresPosibles(c,pokemonEnPos(posPokemonCercano(c,j),j).jugadoresEnRango,j)$}
  [$\Theta(???)$] 
  [Devuelve un iterador a los jugadores que estan esperando para atrapar al pokemon mas cercano a la coordenada ingresada]   

  \InterfazFuncion{indiceRareza}{\In{j}{juego}, \In{p}{Pokemon}}{Nat}
  [$p \in todosLosPokemons(j)$]
  {$res \igobs indiceRareza(p,j)$}
  [$\Theta(???)$] 
  [Devuelve el indice de rareza del pokemon del juego ingresado]

  \InterfazFuncion{cantPokemonesTotales}{\In{j}{juego}}{Nat}
  [true]
  {$res \igobs cantPokemonsTotales(p)$}
  [$\Theta(???)$] 
  [Devuelve la cantidad de pokemones que hay en el juego]

  \InterfazFuncion{cantMismaEspecie}{\In{j}{juego}, \In{p}{Pokemon}}{Nat}
  [true]
  {$res \igobs cantMismaEspecie(p, pokemons(j), j$}
  [$\Theta(???)$] 
  [Devuelve la cantidad de pokemones de la especie ingresada hay en el juego]

  
\end{Interfaz}

\begin{Representacion}
\subsubsection{Representación de Mapa}
	\begin{Estructura}{Juego}[juego]
		\begin{Tupla}[juego]
			\tupItem{pokemones}{diccString$<$String, Nat$>$}%
			\tupItem{jugadores}{conjLineal(jugador)}
			\tupItem{expulsados}{conjLineal(jugador)}
			\tupItem{jugadoresPorID}{Vector($<$itConj(jugador), itColaPrior(jugador)$>$)}
			\tupItem{posicionesPokemons}{DiccAc(coor,$<$itDiccString(pokemon), Bool$>$)}
			\tupItem{mapa}{Mapa}
			\tupItem{pokemonsTotales}{Nat}
		\end{Tupla}
	\end{Estructura}
	
	\begin{Estructura}{Jugador}[jug]
		\begin{Tupla}[jug]
			\tupItem{id}{Nat}%
			\tupItem{posicion}{Coordenada}
			\tupItem{estaConectado}{Bool}
			\tupItem{sanciones}{Nat}
			\tupItem{pokeCapturados}{ConjLineal(itDiccString(pokemon))}
		\end{Tupla}
	\end{Estructura}
	
	\begin{Estructura}{Pokemon}[poke]
		\begin{Tupla}[poke]
			\tupItem{tipo}{String}%
			\tupItem{contador}{Nat}
			\tupItem{jugadoresEnRango}{diccHeap$<$Nat, itConjLineal$>$}
			\tupItem{salvaje}{Bool}
		\end{Tupla}
	\end{Estructura}

\subsubsection{Invariante de Representación}
	\begin{enumerate}
		\item La suma de las cantidades de cada pokemon es igual a pokemonesTotales.
		\item La suma de la cantidad de jugadores y expulsados es igual a la longitud del vector jugadoresPorID.
		\item Para toda coordenada, si esta definida en posicionesPokemons entonces la coordeanda pertenece al mapa.
		\item La posicion de todo jugador que pertenezca al conjunto jugadores y este conectado pertenece al mapa.
		\item Para todo pokemon que exista en pokemons y sea salvaje, el conjunto de jugadores que esta esperando para atraparlo pertenece al conjunto jugadores.
		\item Todo jugador que pertenezca a jugadores, este conectado y este esperando para atrapar, esta incluido en el conjunto de jugadores en rango del pokemon al que quiere atrapar.
		\item Los conjuntos jugadores y expulsados son disjuntos.
	\end{enumerate}
	
	\begin{enumerate}
		\item Checkear con significado de trie
		\item \# e.jugadores + \# e.expulsados = long(e.jugadoresPorID)
		\item ($\forall c : coor$) def?(c, e.posicionesPokemons) $\impluego$ j.posicion $\in$ e.mapa.coordenadas
		\item ($\forall j : jug$) j $\in$ e.jugadores $\land$ j.estaConectado $\impluego$ j.posicion $\in$ e.mapa.coordenadas
		\item ($\forall p : poke$) (def?(p, e.pokemones) $\land$ p.salvaje) $\impluego$ ($\forall it : itJug$) HayMas?(it) $\yluego$ Actual(it) $\in$ p.jugadoresEnRango $\impluego$ Actual(it) $\in$ e.jugadores
		\item ($\forall j : jug$) j $\in$ e.jugadores $\land$ j.estaConectado $\yluego$ estaParaAtrapar(j) $\impluego$ ($\forall p : poke$) def?(p, e.pokemones) $\yluego$ j $\in$ p.jugadoresEnRango
		\item ($\forall j : jug$) (j $\in$ e.jugadores $\impluego$ j $\notin$ e.expulsados) $\lor$ (j $\in$ e.expulsados $\impluego$ j $\notin$ e.jugadores)
		
	\end{enumerate}
	
\subsubsection{Funci\'on de Abstracci\'on}	
	Abs(e): estre - $>$ Jugo {Rep(e)} 
 pGo: Juego tq e.mapa = mapa(pGo) y e.jugadores = jugadores(pGo) yluego \\
 (Para todo j : jugador) j pertenece e.jugadores impluego 
 \\ j.sanciones = sanciones(j, pGo) ((j pertenece expulsados(pGo) y j.sanciones >= 10)\\
 oluego (j.pokesCapturados = pokemones(j,pGo) y j.estaConectado = estaConectad(j,pGo) \\
 y j.estaConectado impluego j.pos = posicion(j,pGo))) y \\
 (Para todo p : pokemon) p pertenece c.pokemones impluego (Para todo j : Jugador) \\
 j pertenece e.jugadores yluego p pertenece pokemones(j,pGo) o [(Para todo c : coord)\\
 c pertenece e.mapa.coordenadas yluego p = pokemonEnPos(c,pGo) y cantMovParaCap(c,pGo)\\
 p.contador]
\end{Representacion}

%%% Sanciones	O(1)
%%% Contador	O(1)
%%% Heap		O(log(EC))
%%% Atrapar		O(PS + PC)
%%% BorrarJ		O(|P| * PC)



%
%  \InterfazFuncion{Tope}{\In{p}{pila($\alpha$)}}{$\alpha$}
%  [$\lnot$vacia?($p$)]
%  {alias($res$ \igobs tope($p$))}
%  [$\Theta(1)$]
%  [devuelve el tope de la pila.]
%  [$res$ es modificable si y solo si $p$ es modificable.]
%
%  \InterfazFuncion{Desapilar}{\Inout{p}{pila($\alpha$)}}{}
%  [$p \igobs p_0$ $\land$ $\lnot$vacia?($p$)]
%  {$p$ \igobs desapilar($p_0$) $\land$ $res$ \igobs tope($p$)}
%  [$\Theta(1)$]
%  [desapila el tope de $p$.]
%
%  \InterfazFuncion{Tamaño}{\In{p}{pila($\alpha$)}}{nat}
%  {$res$ \igobs tamaño($p$)}
%  [$\Theta(1)$]
%  [devuelve la cantidad de elementos apilados en $p$.]
%
