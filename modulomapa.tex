\section{Modulo Mapa}

%TODO descripcion

\begin{Interfaz}

  \textbf{usa}: \tadNombre{Nat, Bool, Coordenada, Conj($\alpha$)}.

  \textbf{se explica con}: \tadNombre{Mapa}.

  \textbf{generos}: \TipoVariable{map}.

  \InterfazFuncion{CrearMapa}{}{map}%
  [true]
  {$res \igobs crearMapa()$}
  [$O(1)$]
  [Crea un nuevo mapa]

  \InterfazFuncion{AgregarCoordenada}{\Inout{m}{map}, \In{c}{coor}}{itConj(coor)}
  [$m \igobs m_0$]
  {$m \igobs agregarCoor(c, m_0)$}
  [$\Theta\left(\displaystyle\sum_{c' \in coordendas(m)}equal(c,c')\right)$]
  [Agrega una coordenada al mapa y devuelve el iterador a la coordenada agregada. Su complejidad es la de agregar un elemento al conjunto lineal.]

  \InterfazFuncion{Coordenadas}{\In{m}{map}}{itConj(coor)}
  [true]
  {$res \igobs coordenadas(m)$}
  [$O(1)$]
  [Devuelve un iterador al conjunto de coordenadas del mapa]
  
  \InterfazFuncion{PosExistente}{\In{c}{coor}, \In{m}{map}}{Bool}
  [true]
  {$res \igobs posExistente(c, m)$}
  [$\Theta\left(\displaystyle\sum_{c' \in coordendas(m)}equal(c,c')\right)$]
  [Devuelve verdadero si la coordenada esta en el conjunto de coordenadas del mapa]
  
  \InterfazFuncion{HayCamino}{\In{c1}{coor}, \In{c2}{coor}, \In{m}{map}}{Bool}
  [$c1 \in coordenadas(m) \land c2 \in coordenadas(m)$]
  {$res \igobs hayCamino(c1, c2, m)$}
  [$\Theta(1)$]
  [Devuelve verdadero si existe un camino entre ambas coordenadas]
  
  \InterfazFuncion{Ancho}{\In{m}{map}}{Nat}
  [$true$]
  {($\forall$ c : coor) c $\in$ coordenadas(m) $\impluego$ longitud(res) $\geq$ longitud(c) }
  [$\Theta(\#coordenadas(m))$]
  [Devuelve el ancho del mapa]
  
  \InterfazFuncion{Largo}{\In{m}{map}}{Nat}
  [$true$]
  {($\forall$ c : coor) c $\in$ coordenadas(m) $\impluego$ latitud(res) $\geq$ latitud(c) }
  [$\Theta(\#coordenadas(m))$]
  [Devuelve el largo del mapa]
  
\end{Interfaz}

\begin{Representacion}
\subsubsection{Representación de Mapa}
	\begin{Estructura}{Mapa}[estr]
		\begin{Tupla}[estr]
			\tupItem{coordenadas}{ConjLineal(coor)}%
			\tupItem{secciones}{DiccMat(coor, Nat)}
		\end{Tupla}
	\end{Estructura}
	
\subsubsection{Invariante de Representación}
	\begin{enumerate}
		\item El ancho del mapa es igual al maximo del primer elemento de las coordenadas
	\end{enumerate}
	\Rep[estr][e]{($\forall c: coor$) c $\in$ e.coordenadas $\impluego$ def?(c,e.secciones)}

\subsubsection{Funci\'on de Abstracci\'on}
	\AbsFc[estr]{Mapa}[e]{($\forall$ m : Mapa) e.coordenadas = coordenadas(m)}

\end{Representacion}

\begin{Algoritmos}

\medskip
	
 \Titulo{Algoritmos del modulo}
  	\medskip
  
\begin{algorithm}[H]{\textbf{iCrearMapa}() $\to$ $res$ : estr}
    	\begin{algorithmic}[1]
			\State $res.coordenadas \gets Vacio()$ \Comment{La complejidad es la de crear el Conjunto Lineal vacio $\Theta(1)$}
			\State res.secciones $\gets$ NULL \Comment O(1)
			
			\medskip
			\Statex \underline{Complejidad:} $\Theta(1)$
    	\end{algorithmic}
\end{algorithm}

\begin{algorithm}[H]{\textbf{iAgregarCoordenada}(\Inout {m}{estr}, \In {c}{coor}) $\to$ $res$ : itConj(coor)}
    	\begin{algorithmic}[1]
            \State  largo $\gets$ Largo(m)   \Comment $\Theta$(Cardinal(m.coordenadas))
            \State  ancho $\gets$ Ancho(m) \Comment $\Theta$(Cardinal(m.coordenadas))
            \State  m.secciones $\gets$ Vacio(largo, ancho) \Comment $\Theta$(largo * ancho)
			\State res $\gets$ Agregar(m.coordenadas, c) \Comment $\Theta\left(\displaystyle\sum_{c' \in coordendas(m)}equal(c,c')\right)$
            \State  seccion $\gets$ 0  \Comment $\Theta(1)$
            \State  itCoor $\gets$ CrearIt(m.coordenadas) \Comment $\Theta(1)$
			 
            \While{$HaySiguiente(itCoor)$}			\Comment $\Theta(Cardinal(m.coordenadas))$
                \State $coord \gets Siguiente(itCoor)  $ \Comment $\Theta(1)$
			 	\State $Avanzar(it)$		\Comment $\Theta(1)$
                \If{$\lnot(Definido?(m.secciones, coord))$} \Comment $\Theta(1)$
                    \State $ DefinirSeccion(m, coord, seccion) $ \Comment $\Theta(Cardinal(m.coordenadas))$ 
                    \State $ seccion \gets seccion + 1 $ \Comment $\Theta(1)$
                \EndIf
			\EndWhile

			\medskip
			\Statex \underline{Complejidad:} $\Theta(Ancho(m) * Largo(m) + Cardinal(m.coordenadas)^2)$
			\Statex \underline{Justificación:} $\Theta(Ancho(m) * Largo(m))$ es mayor o igual que $\Theta(Cardinal(m.coordenadas))$ y el costo de Agregar un elemento a un conjunto lineal. El While tiene complejidad $\Theta(Cardinal(m.coordenadas))$ dentro, y dentro se llama a una funcion con la misma complejidad, luego, por algebra de complejidad, es $\Theta(Ancho(m) * Largo(m) + Cardinal(m.coordenadas)^2)$
    	\end{algorithmic}
\end{algorithm}

\begin{algorithm}[H]{\textbf{iDefinirSeccion}(\Inout {m}{estr}, \In {c}{coor}, \In {i}{Nat})}
    	\begin{algorithmic}[1]
				\If{$\lnot(Definido?(m.secciones, c)) \land PosExistente(c, m)$} \Comment $\Theta\left(\displaystyle\sum_{c' \in coordendas(m)}equal(c,c')\right)$
                \State $Definir(m.secciones, c, i)$ \Comment $\Theta(1)$
                \State $DefinirSeccion(m, CoordenadaArriba(c), i)$ \Comment $\Theta(Cardinal(m.coordenadas))$
                \State $DefinirSeccion(m, CoordenadaALaDerecha(c), i)$ \Comment $\Theta(Cardinal(m.coordenadas))$ 
                \If{$Latitud(c) > 0$} \Comment $\Theta(1)$
                    \State $DefinirSeccion(m, CoordenadaAbajo(c), i)$ \Comment $\Theta(Cardinal(m.coordenadas))$ 
                \EndIf
                \If{$Longitud(c) > 0$} \Comment $\Theta(1)$
                    \State $DefinirSeccion(m, CoordenadaALaIzquierda(c), i)$ \Comment $\Theta(Cardinal(m.coordenadas))$ 
                \EndIf
            \EndIf

			\medskip
			\Statex \underline{Complejidad:} $\Theta(Cardinal(m.coordenadas))$
            \Statex \underline{Justificación} DefinirSeccion se llama a si misma recursivamente recorreindo las coordenadas, en el peor caso, recorre todas las coordenadas una vez, luego su complejidad es $\Theta(4^{Cardinal(m.coordenadas)})$ que se puede simplificar, ya que pertenece a la misma clase. Esta funcion no es cuadratica, ya que usa el diccionario para chequear que no este recorriendo una posicion mas de una vez. 
    	\end{algorithmic}
\end{algorithm}

\begin{algorithm}[H]{\textbf{iCoordenadas}(\In {m}{estr}) $\to$ $res$ : itConj(coor)}
    	\begin{algorithmic}[1]
			\State $res \gets CrearIt(m.coordenadas)$ \Comment {La complejidad es la de crear un iterador a un conjunto lineal $\Theta(1)$}
			
			\medskip
			\Statex \underline{Complejidad:} $\Theta(1)$
    	\end{algorithmic}
\end{algorithm}

\begin{algorithm}[H]{\textbf{iPosExistente}(\In{c}{coor}, \In{m}{estr}) $\to$ $res$ : Bool}
    	\begin{algorithmic}[1]
			\State $res \gets pertenece?(m.coordenadas, c)$ \Comment  $\Theta\left(\displaystyle\sum_{c' \in coordendas(m)}equal(c,c')\right)$
			
			\medskip
			\Statex \underline{Complejidad:} $\Theta\left(\displaystyle\sum_{c' \in coordendas(m)}equal(c,c')\right)$
			\Statex \underline{Justificación:} La complejidad es la fijarse que un elemento pertenezca al conjunto lineal.
    	\end{algorithmic}
\end{algorithm}

\begin{algorithm}[H]{\textbf{iHayCamino}(\In{c1}{coor}, \In{c2}{coor}, \In{m}{estr}) $\to$ $res$ : Bool}
    	\begin{algorithmic}[1]
			\State $res \gets (Definido?(m.secciones, c1) \land Definido?(m.secciones, c2)) \land_L (Significado(m.secciones, c1) = Significado(m.secciones, c2))$ \Comment $\Theta(1)$
			\medskip
			\Statex \underline{Complejidad:} $\Theta(1)$
    	\end{algorithmic}
\end{algorithm}

\begin{algorithm}[H]{\textbf{iAncho}(\In{m}{estr}) $\to$ $res$ : Nat}
    	\begin{algorithmic}[1]
			\State $it \gets CrearIt(m.coordenadas)$ \Comment $\Theta(1)$
			\State $max \gets 0$ \Comment $\Theta(1)$
			\While {HaySiguiente(it)} \Comment $\Theta(Cardinal(m.coordenadas))$
				\If {max $<$ Longitud(Siguiente(it))} \Comment $\Theta(1)$
					\State max $\gets$ Longitud(Siguiente(it)) \Comment $\Theta(1)$	
				\EndIf
				\State $Avanzar(it)$ \Comment $\Theta(1)$				
			\EndWhile
			\medskip
			\Statex \underline{Complejidad:} $\Theta(Cardinal(m.coordenadas))$
    	\end{algorithmic}
\end{algorithm}

\begin{algorithm}[H]{\textbf{iLargo}(\In{m}{estr}) $\to$ $res$ : Nat}
    	\begin{algorithmic}[1]
			\State $it \gets CrearIt(m.coordenadas)$ \Comment $\Theta(1)$
			\State $max \gets 0$ \Comment $\Theta(1)$
			\While {$HaySiguiente(it)$} \Comment $\Theta(Cardinal(m.coordenadas))$
				\If {max $<$ Latitud(Siguiente(it))} \Comment $\Theta(1)$
					\State max $\gets$ Latitud(Siguiente(it)) \Comment $\Theta(1)$					
				\EndIf
				\State $Avanzar(it)$ \Comment $\Theta(1)$				
			\EndWhile
			\medskip
			\Statex \underline{Complejidad:} $\Theta(Cardinal(m.coordenadas))$
    	\end{algorithmic}
\end{algorithm}

\end{Algoritmos}
