\section{Modulo Diccionario Acotado($coordenada$, $\sigma$)}

El modulo Diccionario Acotado provee un diccionario por posiciones en el que se puede definir, borrar, y testear si hay un valor en una posicion en tiempo $O(1)$. 

El principal costo de paga al crear la estructura, dado de cuesta tiempo lineal $ancho$ por $largo$.

\begin{Interfaz}

  \textbf{parametros formales}\hangindent=2\parindent\\
  \parbox{1.7cm}{\textbf{generos}}$coordenada,\sigma$\\

  \textbf{se explica con}: \tadNombre{DiccAcotado$(Nat, \sigma)$},

  \textbf{generos}: \TipoVariable{DiccAc$(coordenada, \sigma)$}.

  \Titulo{Operaciones basicas de tabla}

  \InterfazFuncion{Vacio}{\In{Nat} ancho, \In{Nat} largo}{DiccAc$(coordenada,\sigma)$}%
  [ancho $>$ 0 $\land$ largo $>$ 0]
  {$res$ $\igobs$ vacio(ancho $*$ largo)}%
  [$\Theta(ancho $*$ largo)$]
  [genera un diccionario vacia.]

  \InterfazFuncion{Definir}{\Inout{t}{DiccAc($coordenada,\sigma$)}, \In{c}{$coordenada$}, \In{s}{$\sigma$}}{}
  [$t \igobs t_0$]
  {$t$ $\igobs$ definir($t, c, s$)}
  [$\Theta(1)$]
  [define el significado $s$ en la tabla, en la posicion representada por $c$.]
  [Hay alising, pero no se como explicarlo TODO]

  \InterfazFuncion{Definido?}{\In{t}{tabla($coordenada,\sigma$)}, \In{c}{$coordenada$}}{bool}
  {$res$ $\igobs$ def?($t$, $c$)}
  [$\Theta(1)$]
  [devuelve \texttt{true} si y solo $c$ tiene un valor en la tabla.]

  \InterfazFuncion{Significado}{\In{t}{tabla($coordenada,\sigma$)}, \In{c}{$coordenada$}}{$\sigma$}
  [def?($t$, $c$)]
  {alias($res$ $\igobs$ Significado($t$, $c$))}
  [$\Theta(1)$]
  [devuelve el valor en la posicion $c$ de $t$.]

  \InterfazFuncion{Borrar}{\Inout{t}{tabla($coordenada,\sigma$)}, \In{c}{$coordenada$}}{}
  [$t = t_0$ $\land$ def?($t$, $c$)]
  {$t$ $\igobs$ borrar($t_0, c$)}
  [$\Theta(1)$]
  [elimina el valor en la posicion $c$ en $t$.]

\pagebreak

\subsubsection{Especificacion de las operaciones auxiliares utilizadas en la interfaz}
%%%%%%%%%%%%%%%%%%%% Diccionario Acotado
\begin{tad}{\tadNombre{DiccAcotado(Nat,Significado)}}
	\tadGeneros{diccAc(Nat,Significado)}
	\tadExporta{diccAc(Nat,Significado), 
	generadores, 
	observadores, 
	borrar, 
	claves}
	\tadUsa{\tadNombre{Nat},
	\tadNombre{Bool},
	\tadNombre{Conj(Nat)}},
	\tadIgualdadObservacional{d}{d'}{Dicc(Nat, $\sigma$)}
                         {($\forall$ c: Nat) (enRango(c,d) $\igobs$ enRango(c,d') $\yluego$ \\
                          def?(c,d) $\igobs$ def?(c,d')) $\yluego$ \\
                          def?(c,d) $\impluego$ obtener(c,d) $\igobs$ obtener(c,d')}

	\tadObservadores
	\tadAlinearFunciones{enRango~}{Nat/n, diccAc(Nat,sinificado), Bool}
	
	\tadOperacion{enRango}{Nat, diccAc(Nat,sinificado)}{Bool}{}
    \tadOperacion{def?}{Nat/c, diccAc(Nat,significado)/d}{Bool}{enRango(c,d)}{}
    \tadOperacion{obtener}{Nat/c, diccAc(Nat,significado)/d}{significado}{enRango(c,d) $\yluego$ def?(c,d)}{}

	\tadGeneradores
	\tadAlinearFunciones{definir~}{Nat/c, significado/s, diccAc(Nat,significado)/d}
    \tadOperacion{vacio}{Nat/r}{diccAc(Nat,significado)}{r $>$ 0}{}
    \tadOperacion{definir}{Nat/c, significado/s, diccAc(Nat,significado)/d}{diccAc(Nat,significado)}{enRango(c,d)}

    \tadOtrasOperaciones
    \tadAlinearFunciones{borrar~}{Nat/c, diccAc(Nat,significado)/d, diccAc(Nat,significado)}

	\tadOperacion{borrar}{Nat/c, diccAc(Nat,significado)/d}{diccAc(Nat,significado)}{enRango(c,d) $\yluego$ def?(c,d)}{}
    \tadOperacion{claves}{diccAc(Nat,significado)}{conj(Nat)}{}{}
    
    \tadAxiomas[\paratodo{Nat}{c, k} \paratodo{diccAc(Nat,significado)}{d} \paratodo{significado}{s}]

	\tadAxioma{enRango(c, vacio(r))}{c $<$ r}
	\tadAxioma{def?(c, vacio(r))}{false}
	\tadAxioma{enRango(c, definir(k,s,d))}{enRango(c,d)}
	\tadAxioma{def?(c, definir(k,s,d))}{c = k $\vee$ def?(c,d)}
	\tadAxioma{obtener(c, definir(k,s,d))}{\IF c = k THEN s ELSE obtener(c,d) FI}	
	\tadAxioma{borrar(c, definir(k,s,d))}{\IF c = k THEN {\IF def?(c,d) THEN borrar(c,d) ELSE d FI} ELSE definir(k,s,borrar(c,d)) FI}	
	\tadAxioma{claves(vacio)}{$\emptyset$}
	\tadAxioma{claves(definir(c,s,d))}{Ag(c,claves(d))}
\end{tad}

\end{Interfaz}
