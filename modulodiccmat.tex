\section{Modulo Diccionario Matriz($coor$, $\sigma$)}

El modulo Diccionario Matriz provee un diccionario por posiciones en el que se puede definir, y consultar si hay un valor en una posicion en tiempo $O(copy(\sigma))$. Ademas, se puede borrar en tiempo lineal sobre las dimensiones de la matriz, y obtener un iterador a un conjunto lineal de claves.

El principal costo se paga al crear la estructura o borrar un dato, dado que cuesta tiempo lineal $ancho$ por $largo$.

\begin{Interfaz}

  \textbf{parametros formales}\hangindent=2\parindent\\
  \parbox{1.7cm}{\textbf{generos}}$coor,\sigma$\\
%TODO
  \textbf{se explica con}: \tadNombre{DiccMat$(Nat, Nat, \sigma)$},

  \textbf{generos}: \TipoVariable{diccMat$(coor, \sigma)$}.

  \InterfazFuncion{Vacio}{\In{Nat} largo, \In{Nat} ancho}{diccMat$(coor,\sigma)$}%
  [largo $*$ ancho $>$ 0]
  {$res$ $\igobs$ vacio(largo, ancho)}%
  [$\Theta(ancho $*$ largo)$]
  [Genera un diccionario vacio, de tamaño $ancho*largo$.]

  \InterfazFuncion{Definir}{\Inout{d}{diccMat($coor,\sigma$)}, \In{c}{$coor$}, \In{s}{$\sigma$}}{}
  [$d \igobs d_0$ $\land$ enRango($c_1$, $c_2$, $d$)]
  {$d$ $\igobs$ definir($c_1$, $c_2$, $s$, $d_0$)}
  [$\Theta(copy(s))$]
  [define el significado $s$ en el diccMat, en la posicion representada por $c$.]

  \InterfazFuncion{Definido?}{\In{d}{diccMat($coor,\sigma$)}, \In{c}{$coor$}}{bool}
  [enRango($c_1$, $c_2$, $d$)]
  {$res$ $\igobs$ def?($c_1$, $c_2$, $d$)}
  [$\Theta(1)$]
  [devuelve \texttt{true} si y solo si $c$ tiene un valor en el diccMat.]

  \InterfazFuncion{Significado}{\In{d}{diccMat($coor,\sigma$)}, \In{c}{$coor$}}{$\sigma$}
  [enRango($c_1$, $c_2$, $d$) $\land_L$ def?($c_1$, $c_2$, $d$)] 
  {alias($res$ $\igobs$ obtener($c_1$, $c_2$, $d$)}
  [$\Theta(copy(s))$]
  [Devuelve el valor de $d$ en la posicion $c$.]

  \InterfazFuncion{Borrar}{\Inout{d}{diccMat($coor,\sigma$)}, \In{c}{$coor$}}{}
  [$d = d_0$ $\land$ enRango($c_1$, $c_2$, $d$) $\land_L$ def?($c_1$, $c_2$, $d$)]
  {$d$ $\igobs$ borrar($c_1$, $c_2$, $d_0$)}
  [$\Theta(1)$]
  [Elimina el valor en la posicion $c$ en $d$.]

  \InterfazFuncion{Coordenadas}{\In{d}{diccMat($coor,\sigma$)}}{itConj($coor$)}
  {alias(esPermutacion?(SecuSuby($res$), $claves(d)$))}
  [$\Theta(1)$]
  [Devuelve un iterador al conjunto de claves de $d$.]

  \InterfazFuncion{Ancho}{\In{d}{diccMat($coor,\sigma$)}}{$Nat$}
  {$res$ $\igobs$ ancho($d$)}
  [$\Theta(1)$]
  [Devuelve el ancho de $d$]

  \InterfazFuncion{Largo}{\In{d}{diccMat($coor,\sigma$)}}{$Nat$}
  {$res$ $\igobs$ largo($d$)}
  [$\Theta(1)$]
  [Devuelve el largo de $d$]

\pagebreak

\subsubsection{Especificacion de las operaciones auxiliares utilizadas en la interfaz}
%%%%%%%%%%%%%%%%%%%% Diccionario Matriz
\begin{tad}{\tadNombre{DiccMatriz(Nat, Nat,$\sigma$)}}
	\tadGeneros{diccMat(Nat, Nat, $\sigma$)}
	\tadExporta{diccMat(Nat, Nat, $\sigma$), 
	generadores, 
	observadores, 
	borrar, 
	claves}
	\tadUsa{\tadNombre{Nat},
	\tadNombre{Bool},
	\tadNombre{Conj(tupla(Nat, Nat))}}
	\tadIgualdadObservacional{d}{d'}{DiccMat(Nat, Nat, $\sigma$)}
                         {(ancho(d) $\igobs$ ancho(d') $\land$ largo(d) $\igobs$ largo(d') $\yluego$ \\
                          ($\forall$ x,y: Nat) (def?(x,y,d) $\igobs$ def?(x,y,d')) $\yluego$ \\
                          def?(x,y,d) $\impluego$ obtener(x,y,d) $\igobs$ obtener(x,y,d')}

	\tadObservadores
	\tadAlinearFunciones{obtener~}{diccMat(Nat, Nat, sinificado), Nat, Nat}
	
	\tadOperacion{largo}{diccMat(Nat,Nat,$\sigma$)}{Nat}{}
	\tadOperacion{ancho}{diccMat(Nat,Nat,$\sigma$)}{Nat}{}
    \tadOperacion{def?}{Nat/x, Nat/y, diccMat(Nat, Nat, $\sigma$)/d}{Bool}{enRango(x,y,d)}{}
    \tadOperacion{obtener}{Nat/x, Nat/y, diccMat(Nat,Nat,$\sigma$)/d}{$\sigma$}{enRango(x,y,d) $\yluego$ def?(x,y,d)}{}

	\tadGeneradores
	\tadAlinearFunciones{definir~}{Nat/x, Nat/y, $\sigma$/s, diccMat(Nat,Nat,$\sigma$)/d}
    \tadOperacion{vacio}{Nat/largo,Nat/ancho}{diccMat(Nat,Nat,$\sigma$)}{largo$*$ancho $>$ 0}{}
    \tadOperacion{definir}{Nat/x, Nat/y, $\sigma$/s, diccMat(Nat,Nat,$\sigma$)/d}{diccMat(Nat,Nat,$\sigma$)}{enRango(x,y,d)}

    \tadOtrasOperaciones
    \tadAlinearFunciones{borrar~}{Nat/x, Nat/y, diccMat(Nat,Nat,$\sigma$)/d, diccMat(Nat,Nat,$\sigma$)}

	\tadOperacion{borrar}{Nat/x, Nat/y, diccMat(Nat,Nat,$\sigma$)/d}{diccMat(Nat,Nat,$\sigma$)}{enRango(x,y,d) $\yluego$ def?(x,y,d)}{}
    \tadOperacion{claves}{diccMat(Nat,Nat,$\sigma$)}{conj(tupla(Nat,Nat))}{}{}
    
    \tadTitulo{otras operaciones (no exportadas)}
	\tadAlinearFunciones{enRango~}{Nat/x, Nat/y, diccMat(Nat,Nat,sinificado), Bool}
	\tadOperacion{enRango}{Nat, Nat, diccMat(Nat,Nat,sinificado)}{Bool}{}

    \tadAxiomas[\paratodo{Nat}{x,y,m,n} \paratodo{diccMat(Nat,Nat,$\sigma$)}{d} \paratodo{$\sigma$}{s}]

	\tadAxioma{largo(vacio(m,n))}{m}
	\tadAxioma{ancho(vacio(m,n))}{n}
	\tadAxioma{def?(x,y, vacio(m,n))}{false}
	\tadAxioma{largo(definir(x,y,s,d)))}{largo(d)}
	\tadAxioma{ancho(definir(x,y,s,d)))}{ancho(d)}
	\tadAxioma{def?(x,y, definir(m,n,s,d))}{(x = m $\land$ y = n) $\vee$ def?(x,y,d)}
	\tadAxioma{obtener(x,y, definir(m,n,s,d))}{\IF (x = m $\land$ y = n) THEN s ELSE obtener(x,y,d) FI}	
	\tadAxioma{borrar(x,y, definir(m,n,s,d))}{\IF (x = m $\land$ y = n) THEN {\IF def?(x,y,d) THEN borrar(x,y,d) ELSE d FI} ELSE definir(m,n,s,borrar(x,y,d)) FI}	
	\tadAxioma{claves(vacio)}{$\emptyset$}
	\tadAxioma{claves(definir(x,y,s,d))}{Ag((x,y),claves(d))}
	\tadAxioma{enRango(x,y, d)}{x $<$ largo(d) $\land$ y $<$ ancho(d)}
\end{tad}

\end{Interfaz}

\begin{Representacion}
	\begin{Estructura}{Diccionario Matriz}[dicc]
		\begin{Tupla}[dicc]
			\tupItem{posiciones}{arregloDimensionable de $<bool, itConj(\sigma), \sigma>$}
            \tupItem{claves}{conjLineal(coor)}
			\tupItem{ancho}{Nat}
			\tupItem{largo}{Nat}
		\end{Tupla}
	\end{Estructura}
	
	\tadAlinearFunciones{Rep}{estr}
	\tadAlinearAxiomas{Rep(d)}

	\Rep[dicc][d]{(tam(d.posiciones) $\igobs$ (d.ancho x d.largo)) $\yluego$(($\forall c:$ coor) (c $\in$ d.claves) $\Leftrightarrow$ definido?(d.posiciones, n) $\yluego$ (d.posiciones[n]$_{1}$ $\igobs$ true)) $\yluego$ (($\forall c:$ coor) (c $\in$ d.claves $\Leftarrow$ (Siguiente(d.posiciones[n]$_{2}$) $\igobs$ c)))}%\mbox{}
	
	\tadAlinearFunciones{Abs}{dicc/d}	
	
	\Abs[dicc]{diccMat}[d]{d'}{$d$.claves = claves(d') $\land$ $d$.ancho = ancho(d') $\land$ $d$.largo = largo(d') $\land$ ($\forall c \gets d.claves$) d.posiciones[$c.campo_1 * d.ancho + c.campo_2 $] = siginificado($c_1$, $c_2$, d')}
	

\end{Representacion}



\begin{Algoritmos}

\medskip
	
 \Titulo{Algoritmos del modulo}
  	\medskip
  
\begin{algorithm}[H]{\textbf{iVacio}(\In {l}{Nat},\In {a}{Nat}) $\to$ $res$ : dicc}
    	\begin{algorithmic}[1]
			\State $res.largo \gets l$ \Comment $\Theta(1)$
			\State $res.ancho \gets a$ \Comment $\Theta(1)$
			\State $res.posiciones \gets CrearArreglo(a * l)$ \Comment $\Theta(a*l)$
			\State $res.claves \gets Vacio()$ \Comment $\Theta(1)$
			\medskip
			\Statex \underline{Complejidad:} $\Theta(a*l)$
        \end{algorithmic}
\end{algorithm}

\begin{algorithm}[H]{\textbf{iDefinir}(\Inout {d}{dicc}, \In {c}{coor}, \In {s}{$\sigma$})}
    	\begin{algorithmic}[1]
            \Comment sig : tupla \Comment O(1)
            \Comment sig$_{1}$ $\gets$ true \Comment O(1) 
            \Comment sig$_{3}$ $\gets$ s \Comment $\Theta(copy(s))$
            \If{$\lnot(Definido?(d, Aplanar(d,c)))$}
                \State sig$_{2}$ $\gets$ $AgregarRapido(d.claves, c)$ \Comment $\Theta(copy(s))$
            \Else
                \State sig$_{2}$ $\gets$ d.posiciones[Aplanar(d, c)]$_{2}$ \Comment O(1) 
            \EndIf
            \State $d.posiciones[Aplanar(d, c)] \gets sig$ \Comment $\Theta(copy(s))$
			
			\medskip
			\Statex \underline{Complejidad:} $\Theta(copy(s))$
            \Statex \underline{Justificacion:} $Definido?$ y $Aplanar$ tienen costo $\Theta(1)$, $AgregarRapido$ y $Definir$ tienen costo $\Theta(copy(s))$. Aplicando algebra de ordenes: $\Theta(1)$ + $\Theta(1)$ + $\Theta(copy(s))$  + $\Theta(copy(s))$ = $\Theta(copy(s))$
        \end{algorithmic}
\end{algorithm}

\begin{algorithm}[H]{\textbf{iDefinido?}(\In {d}{dicc}, \In {c}{coor}) $\to$ $res$ : $bool$}
    	\begin{algorithmic}[1]
        \State {$res \gets Definido?(d.posiciones, Aplanar(d, c)) \land_L d.posiciones[Aplanar(d, c)]_1$ } \Comment{Si no esta definido o esta marcado como borrado, se devuelve que no esta definido} $\Theta(1)$
			\medskip
			\Statex \underline{Complejidad:} $\Theta(1)$
            \Statex \underline{Justificacion:} $Aplanar$ tiene costo $\Theta(1)$, luego, como $Definido?$ y consular una posicion de un arreglo tienen costo $\Theta(1)$. Aplicando algebra de ordenes: $\Theta(1)$ + $\Theta(1)$ + $\Theta(1)$ = $\Theta(1)$
        \end{algorithmic}
\end{algorithm}

\begin{algorithm}[H]{\textbf{iSignificado}(\In {d}{dicc}, \In {c}{coor}) $\to$ $res$ : $\sigma$}
    	\begin{algorithmic}[1]
            \State $res \gets d.posiciones[Aplanar(d, c)]$ \Comment $\Theta(1)$
			
			\medskip
			\Statex \underline{Complejidad:} $\Theta(1)$
        \end{algorithmic}
\end{algorithm}

\begin{algorithm}[H]{\textbf{iBorrar}(\Inout {d}{dicc}, \In {c}{coor})}
    	\begin{algorithmic}[1]
            \State EliminarSiguiente(d.posiciones[Aplanar(d, c)]_${2}$) \Comment $\Theta(1)$
            \State $d.posiciones[Aplanar(d, c)] \gets <false, CrearIt(Vacio()), d.posiciones[Aplanar(d, c)]>$ \Comment $\Theta(1)$ 
      \medskip
			\Statex \underline{Complejidad:} $\Theta(1)$
        \end{algorithmic}
\end{algorithm}

\begin{algorithm}[H]{\textbf{iCoordenadas}(\In {d}{dicc}) $\to$ $res$ : $itConj(coor)$}
    	\begin{algorithmic}[1]
            \State $res \gets CrearIt(d.claves)$ \Comment $\Theta(1)$

			\medskip
			\Statex \underline{Complejidad:} $\Theta(1)$
        \end{algorithmic}
\end{algorithm}

\begin{algorithm}[H]{\textbf{iAplanar}(\In {d}{dicc}, \In {c}{coor}) $\to$ $res$ : $nat$}
    	\begin{algorithmic}[1]
			\State $res \gets c.campo_1 * d.ancho + c.campo_2 $ \Comment $\Theta(1)$
			
			\medskip
			\Statex \underline{Complejidad:} $\Theta(1)$
            \Statex \underline{Justificacion:} Son operaciones matematicas de Nat
        \end{algorithmic}
\end{algorithm}

\begin{algorithm}[H]{\textbf{iLargo}(\In{d}{dicc}) $\to$ $res$ : $nat$}
    	\begin{algorithmic}[1]
            \State $res \gets d.largo$ \Comment $\Theta(1)$

			\medskip
			\Statex \underline{Complejidad:} $\Theta(1)$
        \end{algorithmic}
\end{algorithm}

\begin{algorithm}[H]{\textbf{iAncho}(\In{d}{dicc}) $\to$ $res$ : $nat$}
    	\begin{algorithmic}[1]
            \State $res \gets d.ancho$ \Comment $\Theta(1)$

			\medskip
			\Statex \underline{Complejidad:} $\Theta(1)$
        \end{algorithmic}
\end{algorithm}

\end{Algoritmos}
